%\VignetteIndexEntry{Parametric competing risks with simulation based confidence intervals}

\documentclass[a4paper,dvipsnames,twoside,12pt]{report}

% ----------------------------------------------------------------------
% General information for the title page and the page headings
\newcommand{\Title}{Parametric competing risks with simulation based
  confidence intervals}
\newcommand{\Tit}{ParCmpRSim}
\newcommand{\Version}{Version 3}
\newcommand{\Dates}{January 2021}
\newcommand{\Where}{SDCC}
\newcommand{\Homepage}{\url{http://bendixcarstensen.com/}}
\newcommand{\Faculty}{\begin{tabular}{rl}
Bendix Carstensen
  & Steno Diabetes Center Copenhagen, Gentofte, Denmark\\
  & {\small \& Department of Biostatistics,
               University of Copenhagen} \\
  & \href{mailto:b@bxc.dk}{\tt b@bxc.dk} \\
  & \url{http://BendixCarstensen.com} \\[1em]
                      \end{tabular}}

%----------------------------------------------------------------------
% Packages
%\usepackage[inline]{showlabels}
%\usepackage[latin1]{inputenc}
\usepackage[utf8]{inputenc}
\usepackage[T1]{fontenc}
\usepackage[english]{babel}
\usepackage[font=it,labelfont=normalfont]{caption}
\usepackage[colorlinks,urlcolor=blue,linkcolor=red,citecolor=Maroon]{hyperref}
\usepackage[dvipsnames]{xcolor}
\usepackage[super]{nth}
% \usepackage[retainorgcmds]{IEEEtrantools}
\usepackage[noae]{Sweave}
\usepackage[ae,hyper]{Rd}
% \usepackage[noae]{C:/util/R/R-4.2.0/share/texmf/tex/latex/Sweave}
% \usepackage{C:/util/R/R-4.2.0/share/texmf/tex/latex/Rd}
\usepackage{makeidx,floatflt,amsmath,amsfonts,amsbsy,enumitem,dcolumn,needspace}
\usepackage{ifthen,calc,eso-pic,everyshi}
\usepackage{booktabs,longtable,rotating,graphicx,subfig}
\usepackage{pdfpages,verbatim,fancyhdr,datetime,afterpage}
\usepackage[abspath]{currfile}
% \usepackage{times}
\renewcommand{\textfraction}{0.0}
\renewcommand{\topfraction}{1.0}
\renewcommand{\bottomfraction}{1.0}
\renewcommand{\floatpagefraction}{0.9}
\definecolor{blaa}{RGB}{99,99,255}
\DeclareGraphicsExtensions{.png,.pdf,.jpg}
% Make the Sweave output nicer (slightly mor compact)
\DefineVerbatimEnvironment{Sinput}{Verbatim}{fontsize=\small,fontshape=sl,formatcom=\color{BlueViolet}}
\DefineVerbatimEnvironment{Soutput}{Verbatim}{fontsize=\small,formatcom=\color{Sepia},xleftmargin=0em}
\DefineVerbatimEnvironment{Scode}{Verbatim}{fontsize=\small}
\fvset{listparameters={\setlength{\topsep}{-0.1ex}}}
\renewenvironment{Schunk}%
{\renewcommand{\baselinestretch}{0.87} \vspace{\topsep}}%
{\renewcommand{\baselinestretch}{1.00} \vspace{\topsep}}
% \renewenvironment{knitrout}
% {\renewcommand{\baselinestretch}{0.87}}
% {\renewcommand{\baselinestretch}{1.00}}
% This is a file of useful extra commands snatched from
% Michael Hills, David Clayton, Bendix Carstensen & Esa Laara.
%

% Commands to draw observation lines on follow-up diagrams
%
% Horizontal lines
%
\providecommand{\hfail}[1]{\begin{picture}(250,5)
      \put(0,0){\line(1,0){#1}}
      \put(#1,0){\circle*{5}}
   \end{picture}}

\providecommand{\hcens}[1]{\begin{picture}(250,5)
      \put(0,0){\line(1,0){#1}}
      \put(#1,0){\line(0,1){2.5}}
      \put(#1,0){\line(0,-1){2.5}}
   \end{picture}}

%
% Diagonals for Lexis diagrams
%
\providecommand{\dfail}[1]{\begin{picture}(250,250)
      \put(0,0){\line(1,1){#1}}
      \put(#1,#1){\circle*{5}}
   \end{picture}}

\providecommand{\dcens}[1]{\begin{picture}(250,250)
      \put(0,0){\line(1,1){#1}}
%      \put(#1,#1){\line(0,1){2.5}}
%      \put(#1,#1){\line(0,-1){2.5}}
% BxC Changed this to an open circle instead of a line
      \put(#1,#1){\circle{5}}
   \end{picture}}

%
% Horizontal range diagrams
%
\providecommand{\hrange}[1]{\begin{picture}(200,5)
     \put(0,0){\circle*{5}}
     \put(0,0){\line(1,0){#1}}
     \put(0,0){\line(-1,0){#1}}
   \end{picture}}

%
% Tree drawing
%
\providecommand{\Tree}[3]{\setlength{\unitlength}{#1\unitlength}\begin{picture}(0,0)
   \put(0,0){\line(3, 2){1}}
   \put(0,0){\line(3,-2){1}}
   \put(0.81, 0.54){\makebox(0,0)[br]{\footnotesize #2\ }}
   \put(0.81,-0.54){\makebox(0,0)[tr]{\footnotesize #3\ }}
\end{picture}}

\providecommand{\Wtree}[3]{\setlength{\unitlength}{#1\unitlength}\begin{picture}(0,0)
   \put(0,0){\line(1, 1){1}}
   \put(0,0){\line(1,-1){1}}
   \put(0.8,0.8){\makebox(0,0)[br]{\footnotesize #2\ }}
   \put(0.8,-0.8){\makebox(0,0)[tr]{\footnotesize #3\ }}
\end{picture}}

\providecommand{\Ntree}[3]{\setlength{\unitlength}{#1\unitlength}\begin{picture}(0,0)
   \put(0,0){\line(2, 1){1}}
   \put(0,0){\line(2,-1){1}}
   \put(0.8,0.4){\makebox(0,0)[br]{\footnotesize #2\ }}
   \put(0.8,-0.4){\makebox(0,0)[tr]{\footnotesize #3\ }}
\end{picture}}

\providecommand{\Nutree}[3]{\setlength{\unitlength}{#1\unitlength}\begin{picture}(0,0)
   \put(0,0){\line(2, 1){#1}}
   \put(0,0){\line(2,-1){#1}}
   \put(0.8,0.4){\makebox(0,0)[br]{#2\ }}
   \put(0.8,-0.4){\makebox(0,0)[tr]{#3\ }}
\end{picture}}

%
% Tree drawing
%
\providecommand{\tree}[3]{\setlength{\unitlength}{#1}\begin{picture}(0,0)
   \put(0,0){\line(3,2){1}}
   \put(0,0){\line(3,-2){1}}
   \put(0.81,0.54){\makebox(0,0)[br]{\footnotesize #2\ }}
   \put(0.81,-0.54){\makebox(0,0)[tr]{\footnotesize #3\ }}
\end{picture}}

\providecommand{\wtree}[3]{\setlength{\unitlength}{#1}\begin{picture}(0,0)
   \put(0,0){\line(1,1){1}}
   \put(0,0){\line(1,-1){1}}
   \put(0.8,0.8){\makebox(0,0)[br]{\footnotesize #2\ }}
   \put(0.8,-0.8){\makebox(0,0)[tr]{\footnotesize #3\ }}
\end{picture}}

\providecommand{\ntree}[3]{\setlength{\unitlength}{#1}\begin{picture}(0,0)
   \put(0,0){\line(2,1){1}}
   \put(0,0){\line(2,-1){1}}
   \put(0.8,0.4){\makebox(0,0)[br]{\footnotesize #2\ }}
   \put(0.8,-0.4){\makebox(0,0)[tr]{\footnotesize #3\ }}
\end{picture}}

\providecommand{\nutree}[3]{\begin{picture}(0,0)
   \put(0,0){\line(2,1){#1}}
   \put(0,0){\line(2,-1){#1}}
   \put(0.8,0.4){\makebox(0,0)[br]{#2\ }}
   \put(0.8,-0.4){\makebox(0,0)[tr]{#3\ }}
\end{picture}}

%
% Other commands
%
\providecommand{\ip}[2]{\langle #1 \vert #2 \rangle} 
\providecommand{\I}{\text{\rm gI}}
\providecommand{\prob}[0]{\text{\rm Pr}}
\providecommand{\nhy}[0]{_{\oslash}}
\providecommand{\true}[0]{_{\text{\rm \tiny T}}}
\providecommand{\hyp}[0]{_{\text{\rm \tiny H}}}
% \providecommand{\mpydiv}[0]{\stackrel{\textstyle \times}{\div}}
% Changed to slightly smaller symbols
\providecommand{\mpydiv}[0]{\stackrel{\scriptstyle\times}{\scriptstyle\div}}
\providecommand{\mie}[1]{{\it #1}}
\providecommand{\mycircle}[0]{\circle*{5}}
\providecommand{\smcircle}[0]{\circle*{1}}
\providecommand{\corner}[0]{_{\text{\rm \tiny C}}}
\providecommand{\ind}[0]{\hspace{10pt}}
\providecommand{\gap}[0]{\\[5pt]}
\renewcommand{\S}[0]{section~}
\providecommand{\blank}[0]{$\;\,$}
\providecommand{\vone}{\vspace{1cm}}
\providecommand{\ljust}[1]{\multicolumn{1}{l}{#1}}
\providecommand{\cjust}[1]{\multicolumn{1}{c}{#1}}
\providecommand{\transpose}{^{\text{\sf T}}}
\providecommand{\histog}[5]{\rule{1mm}{#1mm}\,\rule{1mm}{#2mm}\,\rule{1mm}{#3mm}\,\rule{1mm}{#4mm}\,\rule{1mm}{#5mm}}
\providecommand{\pmiss}{P_{\mbox{\tiny miss}}}

% Below is BxCs commands inserted

% Only works with hyperref package:
\newcommand{\mailto}[1]{\href{mailto:#1}{\tt #1}}

\providecommand{\bc}{\begin{center}}
\providecommand{\ec}{\end{center}}
\providecommand{\bd}{\begin{description}}
\providecommand{\ed}{\end{description}}
\providecommand{\bi}{\begin{itemize}}
\providecommand{\ei}{\end{itemize}}
\providecommand{\bn}{\begin{equation}}
\providecommand{\en}{\end{equation}}
\providecommand{\be}{\begin{enumerate}}
\providecommand{\ee}{\end{enumerate}}
\providecommand{\bes}{\begin{eqnarray*}}
\providecommand{\ees}{\end{eqnarray*}}

\DeclareMathOperator{\Pp}{P}
\DeclareMathOperator{\pp}{p}
% \providecommand{\p}{{\mathrm p}}
\providecommand{\e}{{\mathrm e}}
\providecommand{\D}{{\mathrm D}}
\providecommand{\dif}{{\,\mathrm d}}
\providecommand{\pmat}[1]{\Pp\!\left\{#1\right\}}
\providecommand{\ptxt}[1]{\Pp\!\left\{\text{#1}\right\}}
\providecommand{\E}{\text{\rm E}}
\providecommand{\V}{\text{\rm V}}
\providecommand{\BLUP}{\text{\rm BLUP}}
\providecommand{\se}{\text{\rm s.e.}}
\providecommand{\sem}{\text{\rm s.e.m.}}
\providecommand{\std}{\text{\rm std}}
\providecommand{\sd}{\text{\rm s.d.}}
\providecommand{\Var}{\text{\rm var}}
\providecommand{\VAR}{\text{\rm var}}
\providecommand{\var}{\text{\rm var}}
\providecommand{\cov}{\text{\rm cov}}
\providecommand{\corr}{\text{\rm corr}}
\providecommand{\mean}{\text{\rm mean}}
\providecommand{\CV}{\text{\rm CV}}
\providecommand{\median}{\text{\rm median}}
\providecommand{\cv}{\text{\rm c.v.}}
\providecommand{\erf}{\text{\rm erf}}
\providecommand{\ef}{\text{\rm ef}}
\providecommand{\SSD}{\text{\rm SSD}}
\providecommand{\SPD}{\text{\rm SPD}}
\providecommand{\odds}{\text{\rm odds}}
\providecommand{\bin}{\text{\rm binom}}
\providecommand{\half}{\frac{1}{2}}
% \providecommand{\td}[0]{\stackrel{\textstyle \times}{\div}}
% Changed to slightly smaller symbols
\providecommand{\td}[0]{\stackrel{\scriptstyle \times}{\scriptstyle \div}}
\providecommand{\dt}[0]{\stackrel{\scriptstyle \div}{\scriptstyle \times}}
\providecommand{\diag}{\text{\rm diag}}
\providecommand{\det}{\text{\rm det}}
\providecommand{\dim}{\text{\rm dim}}
\providecommand{\spcol}{\text{\rm span}}
\providecommand{\logit}{\text{\rm logit}}
% \providecommand{\link}{\text{\rm link}}
\providecommand{\spn}{\text{\rm span}}
\providecommand{\CI}{\text{\rm CI}}
\providecommand{\IP}{\text{\rm IP}}
\providecommand{\OR}{\text{\rm OR}}
\providecommand{\RR}{\text{\rm RR}}
\providecommand{\ER}{\text{\rm ER}}
\providecommand{\EM}{\text{\rm EM}}
\providecommand{\EF}{\text{\rm EF}}
\providecommand{\RD}{\text{\rm RD}}
\providecommand{\AC}{\text{\rm AC}}
\providecommand{\AF}{\text{\rm AF}}
\providecommand{\PAF}{\text{\rm PAF}}
\providecommand{\AR}{\text{\rm AR}}
\providecommand{\CR}{\text{\rm CR}}
\providecommand{\PAR}{\text{\rm PAR}}
\providecommand{\EL}{\text{\rm EL}}
\providecommand{\ERL}{\text{\rm ERL}}
\providecommand{\YLL}{\text{\rm YLL}}
\providecommand{\SD}{\text{\rm SD}}
\providecommand{\SE}{\text{\rm SE}}
\providecommand{\SEM}{\text{\rm SEM}}
\providecommand{\SR}{\text{\rm SR}}
\providecommand{\SMR}{\text{\rm SMR}}
\providecommand{\RSR}{\text{\rm RSR}}
\providecommand{\CMF}{\text{\rm CMF}}
\providecommand{\pvp}{\text{\rm PV$+$}}
\providecommand{\pvn}{\text{\rm PV$-$}}
\providecommand{\R}{{\textsf{\textbf{R}}}}
\providecommand{\sas}{\textsl{\textbf{SAS}}}
\providecommand{\SAS}{\textsl{\textbf{SAS}}}
%\providecommand{\gap}[0]{\\[5pt]}
%\providecommand{\blank}[0]{$\;\,$}
% Conditional independence sign from Philip Dawid
\providecommand{\cip}{\mbox{$\perp\!\!\!\perp$}}

%%% Commands to comment out parts of the text
\providecommand{\GLEM}[1]{}
\providecommand{\FORGETIT}[1]{}
\providecommand{\OMIT}[1]{}

%%% Insert output from program in small text
%%% (requires package verbatim)
\providecommand{\insoutsmall}[1]{
% \small
 \footnotesize
 \renewcommand{\baselinestretch}{0.8}
 \verbatiminput{#1}
 \renewcommand{\baselinestretch}{1.0}
 \normalsize
}
\providecommand{\insout}[1]{
 \scriptsize
 \renewcommand{\baselinestretch}{0.8}
 \verbatiminput{#1}
 \renewcommand{\baselinestretch}{1.0}
 \normalsize
}
\providecommand{\insouttiny}[1]{
\tiny
\renewcommand{\baselinestretch}{0.8}
\verbatiminput{#1}
\renewcommand{\baselinestretch}{1.0}
\normalsize
}

% From Esa:
\providecommand{\T}{\text{\rm \small{T}}}
\providecommand{\id}{\text{\rm id}}
\providecommand{\Dev}{\text{\rm Dev}}
\providecommand{\Bin}{\text{\rm Bin}}
\providecommand{\probit}{\text{\rm probit}}
\providecommand{\cloglog}{\text{\rm cloglog}}

% Special commands to include output from R, Bugs and Stata

\providecommand{\Rin}[2]{
\subsection{\texttt{#1.R}}
#2

\insout{./R/#1.Rout}

}

\providecommand{\Statain}[2]{
\subsection{\texttt{#1.do}}
#2

\insout{./do/#1.log}

}

\providecommand{\Bugsin}[2]{
\subsection{\texttt{#1.bug}}
#2

\insout{./bugs/#1.bug}

}

\newlength{\wdth}
\providecommand{\fxbl}[1]{\settowidth{\wdth}{#1} \makebox[\wdth]{}}

%%% Local Variables:
%%% mode: latex
%%% TeX-master: t
%%% End:


%----------------------------------------------------------------------
% Set up layout of pages
\oddsidemargin 1mm
\evensidemargin 1mm
\topmargin -10mm
\headheight 8mm
\headsep 5mm
\textheight 240mm
\textwidth 165mm
%\footheight 5mm
\footskip 15mm
\renewcommand{\topfraction}{0.9}
\renewcommand{\bottomfraction}{0.9}
\renewcommand{\textfraction}{0.1}
\renewcommand{\floatpagefraction}{0.9}
\renewcommand{\headrulewidth}{0.1pt}
\setcounter{secnumdepth}{2}
\setcounter{tocdepth}{3}

%----------------------------------------------------------------------
% How to insert a figure in a .rnw file
\newcommand{\rwpre}{./graph/gr}
\newcommand{\insfig}[3]{
\begin{figure}[h]
  \centering
  \includegraphics[width=#2\textwidth]{\rwpre-#1}
% \caption{#3}
  \caption{#3\hfill\mbox{\footnotesize \textrm{\tt \rwpre-#1}}}
  \label{fig:#1}
% \afterpage{\clearpage}
\end{figure}}
\newcommand{\linput}[1]{
% \clearpage 
\afterpage{\hfill \ldots now input from \texttt{#1.tex}\\} 
\fancyfoot[OR,EL]{\footnotesize \texttt{#1.tex}} 
\input{#1}}

%----------------------------------------------------------------------
% Here is the document starting with the titlepage
\begin{document}

%----------------------------------------------------------------------
% The title page
\setcounter{page}{1}
\pagenumbering{roman}
\pagestyle{plain}
\thispagestyle{empty}
% \vspace*{0.05\textheight}
\flushright
% The blank below here is necessary in order not to muck up the
% linespacing in title if it has more than 2 lines
{\Huge \bfseries \Title

}\ \\[-1.5ex]
\noindent\textcolor{blaa}{\rule[-1ex]{\textwidth}{5pt}}\\[2.5ex]
\large
\Where \\
\Dates \\
\Homepage \\
\Version \\[1em]
\normalsize
Compiled \today,\ \currenttime\\
from: \texttt{\currfileabspath}\\[1em]
% \input{wordcount}
\normalsize
\vfill
\Faculty
% End of titlepage
\newpage

%----------------------------------------------------------------------
% Table of contents
\tableofcontents
% \listoftables
% \listoffigures
\clearpage
% \begingroup
% \let\clearpage\relax
% \listoftables
% \listoffigures
% \endgroup

%----------------------------------------------------------------------
% General text layout
\raggedright
\parindent 1em
\parskip 0ex
\cleardoublepage

%----------------------------------------------------------------------
% General page style
\pagenumbering{arabic}
\setcounter{page}{1}
\pagestyle{fancy}
\renewcommand{\chaptermark}[1]{\markboth{\textsl{#1}}{}}
\renewcommand{\sectionmark}[1]{\markright{\thesection\ \textsl{#1}}{}}
\fancyhead[EL]{\bf \thepage \quad \rm \rightmark}
\fancyhead[ER]{\rm \Tit}
\fancyhead[OL]{\rm \leftmark}
\fancyhead[OR]{\rm \rightmark \quad \bf \thepage}
\fancyfoot{}

\renewcommand{\rwpre}{./crisk}
\chapter{Competing risks in practice}

The concept of competing risks is one where persons in a given state,
'alive', say,  er subject to a number of different causes of deaths,
'cause1', 'cause2' etc. Causes of death are required to be exhaustive
and mutually exclusive. In situations where the causes are not causes
of death but other events, it is implicit that we only consider the
first occurrence of an event from the state 'alive', and ignore what
occurs after.

The likelihood for observations from a competing risk scenario is a
function of the cause-specific transition rates, and is \emph{product}
of the likelihoods that would emerge if we considered each cause the
only one. Thus analysis is in principle straight forward; just estimate
a model for each of the cause-specific rates. These will together
form a complete model for the competing risks problem.

If the cause-specific rates are all we want to assess then we will be done.

But most often we would like to have estimates of the cumulative
risks, that is the probability of dying from a specific cause before a
given time as function of time. Each of these are functions of
\emph{all} rates. Specifically, if the cause-specific rates are
$\lambda_c(t)$, then:
\[
   R_c(t) = \int_0^t \lambda_c(s) \exp\left(-\int_0^s 
            \sum_j \lambda_j(u) \dif u \right) \dif s
\]
Even if we from the modeling of the $\lambda$s have standard errors of
$\log(\lambda_c)$ the standard errors of $R_c$s will be analytically
intractable from these. 

The only viable way to get confidence intervals for the cumulative
risks, $R_c$, is by calculation of the rates $\lambda(t)$ by sampling
from the posterior distribution of the parameters in the models for
$\log(\lambda(s))$, and computing the integrals numerically for each
simulated sample. 

The simulation approach also allows calculation of confidence
intervals for sums of the cumulative risks, $R_1(t)+R_2(t)$, for
example, which will be needed if we want to show stacked cumulative
risks.

Finally, it will also allow calculation of standard errors of sojourn
times in each of the states 'alive' and 'cause1', 'cause2'. While the
latter two may not be of direct interest, then \texttt{differences}
between such sojourn times between different groups can be
interpreted as years of life lost to each cause between groups.  

\section{Example data}

As an illustrative data example we use the (fake) diabetes register
data; we set up the Lexis object, cut the follow-up time at dates of
OAD, resp Ins:
\begin{Schunk}
\begin{Sinput}
> library(Epi)
> library(popEpi)
> data(DMlate)
> Ldm <- Lexis(entry = list( per = dodm,
+                            age = dodm-dobth, 
+                            tfd = 0 ),
+               exit = list( per = dox ),
+        exit.status = factor( !is.na(dodth), labels = c("DM","Dead") ),
+               data = DMlate )
\end{Sinput}
\begin{Soutput}
NOTE: entry.status has been set to "DM" for all.
NOTE: Dropping  4  rows with duration of follow up < tol
\end{Soutput}
\begin{Sinput}
> summary(Ldm, t = T)
\end{Sinput}
\begin{Soutput}
Transitions:
     To
From   DM Dead  Records:  Events: Risk time:  Persons:
  DM 7497 2499      9996     2499   54273.27      9996

Timescales:
per age tfd 
 ""  ""  "" 
\end{Soutput}
\begin{Sinput}
> Mdm <- mcutLexis( Ldm,
+                    wh = c('dooad','doins'),
+            new.states = c('OAD','Ins'),
+             precursor = 'DM',
+            seq.states = FALSE,
+                  ties = TRUE )
\end{Sinput}
\begin{Soutput}
NOTE: 15 records with tied events times resolved (adding 0.01 random uniform),
      so results are only reproducible if the random number seed was set.
\end{Soutput}
\begin{Sinput}
> summary( Mdm )
\end{Sinput}
\begin{Soutput}
Transitions:
     To
From        DM Dead  OAD  Ins Ins+OAD  Records:  Events: Risk time:  Persons:
  DM      2830 1056 2957  689       0      7532     4702   22920.32      7532
  OAD        0  992 3327    0    1005      5324     1997   22965.25      5324
  Ins        0  152    0  462     172       786      324    3883.07       786
  Ins+OAD    0  299    0    0     878      1177      299    4504.62      1177
  Sum     2830 2499 6284 1151    2055     14819     7322   54273.27      9996
\end{Soutput}
\end{Schunk}
We initially split the FU before drug inception in intervals of
1/12 year, creating a \texttt{Lexis} object for a competing risks
situation with three possible event types:
\begin{Schunk}
\begin{Sinput}
> Sdm <- splitMulti(factorize(subset(Mdm, lex.Cst == "DM")),
+                   tfd = seq(0, 20, 1/12))
\end{Sinput}
\begin{Soutput}
NOTE: lex.Cst and lex.Xst now have levels:
 DM Dead OAD Ins 
\end{Soutput}
\end{Schunk}
We can illustrate the follow-up in the full data set and in the restricted
\begin{Schunk}
\begin{Sinput}
> boxes(Mdm, boxpos = list(x = c(15, 50, 15, 85, 85),
+                          y = c(85, 50, 15, 85, 15)), 
+           scale.R = 100, 
+           show.BE = TRUE)
\end{Sinput}
\end{Schunk}
\insfig{boxes5}{1.0}{The transitions in the multistate model, where
    follow-up is extended also after beginning of first drug
    exposure. Rates in brackets are per 100 PY.}%
\begin{Schunk}
\begin{Sinput}
> boxes( Relevel(Sdm, c(1, 4, 2, 3)), 
+        boxpos  = list(x = c(15, 85, 80, 15),
+                       y = c(85, 85, 20, 15)),
+        scale.R = 100, 
+        show.BE = TRUE )
\end{Sinput}
\end{Schunk}
\insfig{boxes4}{1.0}{The transitions in the competing risks model,
  where follow-up is stopped at drug exposure. By that token only the
  \texttt{DM} state has person-years; a characteristic of a competing risks
  situation.}

\section{Models for rates}

Now that we have set up a dataset with three competing events, we can
model the cause-specific rates separately by time from diagnosis as
the only underlying time scale. Note that we only need to specify the
\texttt{to=} argument because there is only one possible \texttt{from}
for each \texttt{to} (incidentally the same for all \texttt{to}
states, namely \texttt{DM}):
\begin{Schunk}
\begin{Sinput}
> mD <- gam.Lexis(Sdm, ~ s(tfd, k = 5), to = 'Dead')
\end{Sinput}
\begin{Soutput}
mgcv::gam Poisson analysis of Lexis object Sdm with log link:
Rates for the transition: DM->Dead
\end{Soutput}
\begin{Sinput}
> mO <- gam.Lexis(Sdm, ~ s(tfd, k = 5), to = 'OAD' )
\end{Sinput}
\begin{Soutput}
mgcv::gam Poisson analysis of Lexis object Sdm with log link:
Rates for the transition: DM->OAD
\end{Soutput}
\begin{Sinput}
> mI <- gam.Lexis(Sdm, ~ s(tfd, k = 5), to = 'Ins' )
\end{Sinput}
\begin{Soutput}
mgcv::gam Poisson analysis of Lexis object Sdm with log link:
Rates for the transition: DM->Ins
\end{Soutput}
\end{Schunk}
With these models fitted we can compute the rates, cumulative rates
and the cumulative risks an sojourn times in states using the usual
formulae. First we compute the rates in intervals of length 1/100
years. Note that these models only have time since diagnosis as
covariates, so they are the counterpart of Nelson-Aalen estimates,
albeit in a biologically more meaningful guise.

The points where we compute the predicted rates are midpoints of
intervals of length 1/100 year. These points are unrelated to the
follow-up intervals in which we split the data---they were 1 month
intervals, here we use 1/100 year (about 3.7 days): 
\begin{Schunk}
\begin{Sinput}
> int <- 1/100
> nd <- data.frame( tfd = seq(int,10,int)-int/2 ) # not the same as the split, 
>                                                 # and totally unrelated to it
> rownames(nd) <- nd$tfd
> str(nd)
\end{Sinput}
\begin{Soutput}
'data.frame':	1000 obs. of  1 variable:
 $ tfd: num  0.005 0.015 0.025 0.035 0.045 0.055 0.065 0.075 0.085 0.095 ...
\end{Soutput}
\end{Schunk}
With this we can show the rates as a function of the time since diagnosis: 
\begin{Schunk}
\begin{Sinput}
> matshade(nd$tfd, cbind(ci.pred(mD, nd),
+                        ci.pred(mI, nd),
+                        ci.pred(mO, nd))*1000, 
+          ylim = c(0.02,500), yaxt = "n",
+          ylab = "Rates per 1000 PY", 
+          xlab = "Time since DM diagnosis (years)",
+          col = c("black","red","blue"), log = "y", lwd = 3, plot = TRUE)
> axis(side = 2, at = ll<-outer(c(1,2,5),-2:3,function(x,y) x*10^y),
+                labels = formatC(ll,digits = 4), las = 1)
> axis(side = 2, at = ll<-outer(c(1.5,2:9),-2:3,function(x,y) x*10^y),
+                labels = NA, tcl = -0.3)
> text(0, 0.5*0.6^c(1,2,0), 
+      c("Dead","Ins","OAD"),
+      col = c("black","red","blue"), adj = 0)
\end{Sinput}
\end{Schunk}
\insfig{rates}{0.8}{Estimated rates from the \textrm{\tt DM} state,
  estimates are from \textrm{\tt gam} models fitted to data split in 1
  month intervals (1/12 year, that is). Rates of \textrm{\tt OAD} is
  in the vicinity of 0.1/year, and mortality about half of this. Rates
  of insulin start among persons on no other drug are beginning high
  decreasing to about 4 year and then have a peak at 8 years.}  
Note that the graph in figure \ref{fig:rates} is not normally shown in
analyses of competing risks; the competing cause-specific rates are
hardly ever shown. I suspect that this is frequently because they are
often modeled by a Cox model and so are buried in the model.

\section{Cumulative rates and risks}

For the calculation of the cumulative rates and state probabilities,
we need just the rates without CIs: 
\begin{Schunk}
\begin{Sinput}
> # rates at midpoints
> lD <- ci.pred( mD, nd )[,1]
> lI <- ci.pred( mI, nd )[,1]
> lO <- ci.pred( mO, nd )[,1]
> # cumulative rates and survival fuction at right border of the intervals
> LD <- cumsum(lD) * int
> LI <- cumsum(lI) * int
> LO <- cumsum(lO) * int
> Sv <- exp( -LD - LI - LO )
> # but when integrating to get the cumulative risks we use the average
> # of the survival function at the two endpoints (adding 1 as the first)
> mp <- function(x) x - diff(c(1, x)) / 2
> rD <- cumsum(lD * mp(Sv)) * int
> rI <- cumsum(lI * mp(Sv)) * int
> rO <- cumsum(lO * mp(Sv)) * int
\end{Sinput}
\end{Schunk}
Now we have the cumulative risks for the three causes and the
survival, computed at the end of each of the intervals, at any time
point the sum of the 3 cumulative risks and the survival should be 1:
\begin{Schunk}
\begin{Sinput}
> summary(rD + rI +rO + Sv)
\end{Sinput}
\begin{Soutput}
   Min. 1st Qu.  Median    Mean 3rd Qu.    Max. 
      1       1       1       1       1       1 
\end{Soutput}
\begin{Sinput}
> oo <- options(digits = 20)
> cbind(summary(Sv + rD + rI + rO))
\end{Sinput}
\begin{Soutput}
                         [,1]
Min.    1.0000000534682520481
1st Qu. 1.0000010095665299303
Median  1.0000010154411886898
Mean    1.0000009973166215094
3rd Qu. 1.0000010316432050850
Max.    1.0000010385709348082
\end{Soutput}
\begin{Sinput}
> options(oo)
\end{Sinput}
\end{Schunk}
We can then plot the 3 cumulative risk functions together:
\begin{Schunk}
\begin{Sinput}
> zz <- mat2pol(cbind(rD,rI,rO,Sv), x = nd$tfd, 
+               xlim = c(0,10), xaxs = "i", yaxs = "i", las = 1,
+               xlab = "Time since DM diagnosis (years)", 
+               ylab = "Probability",
+                col =  c("black","red","blue","forestgreen") )
> mm <- t(apply(zz,1,mid<-function(x) x[-1]-diff(x)/2))
> text( 9, mm[900,], c("Dead","Ins","OAD","DM"), col = "white" )
> box(col = "white",lwd = 3)
\end{Sinput}
\end{Schunk}
\insfig{stack}{1.0}{Probabilities of being in the 4 different states
  as a function of time since diagnosis. Note that \texttt{OAD} means
  that OAD was initiated first, and similarly for \texttt{Ins}. We are
  not concerned about what occur after these events.  \texttt{Dead}
  means dead without being on any drug.}

\chapter{Confidence intervals}

We want confidence intervals for each of the 4 cumulative risks, but
we may also be interested in confidence intervals for \emph{sums} of
any subset of the cumulative risks, corresponding to the borders
between the colours in figure \ref{fig:stack}. If we only had two
competing risks (and hence three states) the latter would not be an
issue, because the sum of any two cumulative risks will be 1 minus the
cumulative risk of the remainder, so we could get away with the
confidence intervals for the single cumulative risks. This is the
reason we have chosen an example with 3 competing risks and not just
2; we then have 4 probabilities to sum in different order.

A short look at the formulae for cumulative risks will reveal that
analytic approximation to the standard error of these probabilities
(or some transform of them) is not really a viable way to
go. Particularly if we also want confidence intervals of sums of the
state probabilities as those shown in stacked plots.

So in practice, if we want confidence intervals not only for the state
probabilities, but also for any sum of subsets of them we would want a
large number of simulated copies of the cumulative risks, each copy of
the same structure as the one we just extracted from the model.

Moreover, we might also want confidence intervals for sojourn times
(i.e. time spent) in each state up to a given time, which would
come almost for free from the simulation approach.

This means that we must devise a method to make a prediction not from
the estimated model, but where we instead of the model parameters use
a sample from the posterior distribution of the estimated parameters.
Here the posterior distribution of the parameters is taken to be the
multivariate normal distribution with mean equal to the vector of
parameter estimates and variance-covariance matrix equal to the
estimated variance-covariance matrix of the parameters.

Precisely this approach is implemented in \texttt{ci.lin} via the
\texttt{sample} argument; we can get a predicted value from a given
prediction data frame just as from \texttt{ci.pred}
resp. \texttt{ci.exp}; here is an indication of different ways of
getting predicted values of the cause-specific rates:
\begin{Schunk}
\begin{Sinput}
> head(cbind(ci.pred(mI,nd),     ci.exp(mI,nd)            ))
\end{Sinput}
\begin{Soutput}
       Estimate      2.5%     97.5% exp(Est.)      2.5%     97.5%
0.005 0.3355221 0.3050046 0.3690931 0.3355221 0.3050046 0.3690931
0.015 0.3240042 0.2950104 0.3558474 0.3240042 0.2950104 0.3558474
0.025 0.3128817 0.2853291 0.3430948 0.3128817 0.2853291 0.3430948
0.035 0.3021411 0.2759506 0.3308172 0.3021411 0.2759506 0.3308172
0.045 0.2917693 0.2668654 0.3189974 0.2917693 0.2668654 0.3189974
0.055 0.2817538 0.2580639 0.3076184 0.2817538 0.2580639 0.3076184
\end{Soutput}
\begin{Sinput}
> head(cbind(ci.pred(mI,nd), exp(ci.lin(mI,nd)[,c(1,5:6)])))
\end{Sinput}
\begin{Soutput}
       Estimate      2.5%     97.5%  Estimate      2.5%     97.5%
0.005 0.3355221 0.3050046 0.3690931 0.3355221 0.3050046 0.3690931
0.015 0.3240042 0.2950104 0.3558474 0.3240042 0.2950104 0.3558474
0.025 0.3128817 0.2853291 0.3430948 0.3128817 0.2853291 0.3430948
0.035 0.3021411 0.2759506 0.3308172 0.3021411 0.2759506 0.3308172
0.045 0.2917693 0.2668654 0.3189974 0.2917693 0.2668654 0.3189974
0.055 0.2817538 0.2580639 0.3076184 0.2817538 0.2580639 0.3076184
\end{Soutput}
\end{Schunk}
Here is an illustration of the prediction with model based confidence
intervals for the rates, alongside predictions based on samples from
the posterior distribution of the parameters in the model:
\begin{Schunk}
\begin{Sinput}
> str(ci.lin(mI, nd, sample = 4))
\end{Sinput}
\begin{Soutput}
 num [1:1000, 1:4] -1.14 -1.18 -1.21 -1.24 -1.28 ...
 - attr(*, "dimnames")=List of 2
  ..$ : chr [1:1000] "0.005" "0.015" "0.025" "0.035" ...
  ..$ : NULL
\end{Soutput}
\begin{Sinput}
> head(cbind(ci.pred(mI,nd), exp(ci.lin(mI, nd, sample = 4))))
\end{Sinput}
\begin{Soutput}
       Estimate      2.5%     97.5%                                        
0.005 0.3355221 0.3050046 0.3690931 0.3767340 0.3368546 0.3618715 0.3346903
0.015 0.3240042 0.2950104 0.3558474 0.3633041 0.3245842 0.3484644 0.3233786
0.025 0.3128817 0.2853291 0.3430948 0.3503529 0.3127608 0.3355540 0.3124492
0.035 0.3021411 0.2759506 0.3308172 0.3378636 0.3013682 0.3231222 0.3018893
0.045 0.2917693 0.2668654 0.3189974 0.3258196 0.2903907 0.3111510 0.2916864
0.055 0.2817538 0.2580639 0.3076184 0.3142052 0.2798133 0.2996236 0.2818285
\end{Soutput}
\end{Schunk}
The simulation is taking place at the parameter level and the
transformation to survival and cumulative risks is simply a function
applied to every simulated set of rates.

\section{Joint models for several transitions}

Note that we are implicitly assuming that the transitions are being
modeled separately. If some transitions are modeled jointly---for
example assuming that the rates of \texttt{OAD} and \texttt{Ins} are
proportional as functions of time since entry, using one model---we
are in trouble, because we then need one sample from the posterior
generating two predictions, one for each of the transitions modeled
together. Moreover the model will have to be a model fitted to a
\texttt{stack.Lexis} object, so a little more complicated to work with. 

A simple way to program would be to reset the seed to the same value
before simulating with different values of \texttt{nd}, this is what
is intended to be implemented, but is not yet. This is mainly the
complication of having different prediction frames for different risks
in this case. 

Finally, it is not a very urgent need, since the situation where you
want common parameters for different rates out of a common state is
quite rare.

\section{Simulation based confidence intervals}

These ideas have been implemented in the function \texttt{ci.Crisk}
(\texttt{c}onfidence \texttt{i}ntervals for \texttt{C}umulative
\texttt{risk}s) in the \texttt{Epi} package:
We can now run the function using the model objects for the three
competing events, using a common prediction data frame, \texttt{nd}
for the rates:
\begin{Schunk}
\begin{Sinput}
> system.time(
+ res <- ci.Crisk(list(OAD = mO, 
+                      Ins = mI, 
+                     Dead = mD),
+                             nd = data.frame(tfd = (1:1000-0.5)/100),
+                             nB = 1000,
+                           perm = 4:1))
\end{Sinput}
\begin{Soutput}
Times are assumed to be in the column tfd at equal distances of 0.01 
   user  system elapsed 
 12.326   0.669  12.477 
\end{Soutput}
\begin{Sinput}
> str(res)
\end{Sinput}
\begin{Soutput}
List of 3
 $ Crisk: num [1:1001, 1:4, 1:3] 1 0.991 0.983 0.975 0.967 ...
  ..- attr(*, "dimnames")=List of 3
  .. ..$ time : chr [1:1001] "0" "1" "2" "3" ...
  .. ..$ cause: chr [1:4] "Surv" "OAD" "Ins" "Dead"
  .. ..$      : chr [1:3] "50%" "2.5%" "97.5%"
 $ Srisk: num [1:1001, 1:3, 1:3] 0 0.000696 0.001385 0.002065 0.002738 ...
  ..- attr(*, "dimnames")=List of 3
  .. ..$ time : chr [1:1001] "0" "1" "2" "3" ...
  .. ..$ cause: chr [1:3] "Dead" "Dead+Ins" "Dead+Ins+OAD"
  .. ..$      : chr [1:3] "50%" "2.5%" "97.5%"
 $ Stime: num [1:1000, 1:4, 1:3] 0.00996 0.01983 0.02962 0.03933 0.04896 ...
  ..- attr(*, "dimnames")=List of 3
  .. ..$      : chr [1:1000] "1" "2" "3" "4" ...
  .. ..$ cause: chr [1:4] "Surv" "OAD" "Ins" "Dead"
  .. ..$      : chr [1:3] "50%" "2.5%" "97.5%"
 - attr(*, "int")= num 0.01
\end{Soutput}
\end{Schunk}
As we see, the returned object (\texttt{res}) is a list of length 3,
each element a 3-way arrays.  The three components of \texttt{res}
represent
\begin{itemize}
\item \texttt{Crisk} \texttt{C}umulative \texttt{risk}s for each state
\item \texttt{Srisk} \texttt{S}tacked cumulative \texttt{risk}s
  across states  
\item \texttt{Stime} \texttt{S}ojourn
  \texttt{time} for each state, truncated at each point of the time
  dimension, hence there is no 0 in the time dimension 
\end{itemize}

The first dimension of each is time as interval \emph{number},
starting with 0, and corresponding to endpoints of intervals of length
\texttt{int}. The second dimension is states (or combinations
thereof). The last dimension of the arrays is the type of statistic;
\texttt{50\%} the median of the samples, and the bootstrap intervals
as indicated.

The argument \texttt{perm} governs in which order the state
probabilities are stacked in the \texttt{Srisk} element of the
returned list, the default is the states in the
order given in the list of models in the first argument to
\texttt{ci.Crisk} followed by the survival.

If we want the bootstrap samples to make other calculations we can ask
the function to return the bootstrap samples of the rates by using the
argument \texttt{sim.res='rates'} (defaults to \texttt{'none'}):
\begin{Schunk}
\begin{Sinput}
> system.time(
+ rsm <- ci.Crisk(list(OAD = mO, 
+                      Ins = mI, 
+                     Dead = mD),
+                             nd = data.frame(tfd = (1:1000-0.5)/100),
+                             nB = 2000,
+                        sim.res = 'rates'))
\end{Sinput}
\begin{Soutput}
Times are assumed to be in the column tfd at equal distances of 0.01 
   user  system elapsed 
  0.461   0.500   0.320 
\end{Soutput}
\begin{Sinput}
> str(rsm) 
\end{Sinput}
\begin{Soutput}
 num [1:1000, 1:3, 1:2000] 0.453 0.449 0.444 0.44 0.436 ...
 - attr(*, "dimnames")=List of 3
  ..$ time: chr [1:1000] "1" "2" "3" "4" ...
  ..$ mod : chr [1:3] "OAD" "Ins" "Dead"
  ..$ sim : chr [1:2000] "1" "2" "3" "4" ...
 - attr(*, "int")= num 0.01
\end{Soutput}
\end{Schunk}
This is bootstrap samples of the rates evaluated at the 1000 midpoints
of intervals.

Alternatively we can get the bootstrap samples of the cumulative risks
by setting \texttt{sim.res='crisk'}:
\begin{Schunk}
\begin{Sinput}
> system.time(
+ csm <- ci.Crisk(list(OAD = mO, 
+                      Ins = mI, 
+                     Dead = mD),
+                             nd = data.frame(tfd = (1:1000-0.5)/100),
+                             nB = 2000,
+                        sim.res = 'crisk'))
\end{Sinput}
\begin{Soutput}
Times are assumed to be in the column tfd at equal distances of 0.01 
   user  system elapsed 
  6.192   0.554   6.109 
\end{Soutput}
\begin{Sinput}
> str(csm) 
\end{Sinput}
\begin{Soutput}
 num [1:1001, 1:4, 1:2000] 1 0.991 0.982 0.974 0.966 ...
 - attr(*, "dimnames")=List of 3
  ..$ time : chr [1:1001] "0" "1" "2" "3" ...
  ..$ cause: chr [1:4] "Surv" "OAD" "Ins" "Dead"
  ..$ sim  : chr [1:2000] "1" "2" "3" "4" ...
 - attr(*, "int")= num 0.01
\end{Soutput}
\end{Schunk}
This is the cumulative risks evaluated at the 1001 endpoints of the
1000 intervals, and also includes the survival probability in the
first slot of the \nth{1} dimension of \texttt{rsm}.

In both cases, the first slot of the \nth{3} dimension, \texttt{sim},
is the rates, resp. cumulative risks from the model.

\section{Simulated confidence intervals for rates}

In figure \ref{fig:rates} we showed the rates with confidence
intervals from the model. But in \texttt{rsm} we have 2000
(parametric) bootstrap samples of the occurrence rates, so we can
derive the bootstrap medians and the bootstrap c.i.---remember that
the first slice of the \nth{3} dimension is the model estimates that
should not enter the calculations. We use the function \texttt{mnqt}
to compute the model estimate and the mean, median and quantiles of
the simulated values. 
\begin{Schunk}
\begin{Sinput}
> Brates <- aperm(apply(rsm, 1:2, Epi:::mnqt), c(2,3,1))
> str(Brates)
\end{Sinput}
\begin{Soutput}
 num [1:1000, 1:3, 1:3] 0.458 0.454 0.449 0.445 0.441 ...
 - attr(*, "dimnames")=List of 3
  ..$ time: chr [1:1000] "1" "2" "3" "4" ...
  ..$ mod : chr [1:3] "OAD" "Ins" "Dead"
  ..$     : chr [1:3] "50%" "2.5%" "97.5%"
\end{Soutput}
\end{Schunk}
Then we can plot the bootstrap estimates on top of the estimates based
on the normal approximation to distribution of the parameters. They
are not surprisingly in close agreement since they are both based on
an assumption of normality of the parameters on the log-rate scale:
\begin{Schunk}
\begin{Sinput}
> matshade(nd$tfd, cbind(ci.pred(mD, nd),
+                        ci.pred(mI, nd),
+                        ci.pred(mO, nd))*1000, 
+          ylim = c(0.1,500), yaxt = "n",
+          ylab = "Rates per 1000 PY", 
+          xlab = "Time since DM diagnosis (years)",
+          col = c("black","red","blue"), log = "y", lwd = 3, plot = TRUE)
> matlines(nd$tfd, cbind(Brates[,"Dead",],
+                        Brates[,"Ins" ,],
+                        Brates[,"OAD" ,])*1000,
+          col = c("white","black","black"), lty = 3, lwd=c(3,1,1))
> axis(side = 2, at = ll<-outer(c(1,2,5),-2:3,function(x,y) x*10^y),
+                labels = formatC(ll,digits = 4), las = 1)
> axis(side = 2, at = ll<-outer(c(1.5,2:9),-2:3,function(x,y) x*10^y),
+                labels = NA, tcl = -0.3)
> text(0, 0.5*0.6^c(1,2,0), 
+      c("Dead","Ins","OAD"),
+      col = c("black","red","blue"), adj = 0)
\end{Sinput}
\end{Schunk}
\insfig{rates-ci}{0.8}{Estimated rates from the \textrm{\tt DM} state,
  estimates are from \textrm{\tt gam} models fitted to data split in 1
  month intervals (1/12 year, that is). The white dotted curves are
  the bootstrap medians, black dotted curves are the bootstrap 95\%
  c.i.s.}   

\section{Confidence intervals for cumulative risks}

In the \texttt{Crisk} component of \texttt{res} we have the
cumulative risks as functions of of time, with bootstrap confidence
intervals, so we can immediately plot the three cumulative risks:
\begin{Schunk}
\begin{Sinput}
> matshade(c(0,nd$tfd+1/200),
+          cbind(res$Crisk[,"Dead",],
+                res$Crisk[,"Ins" ,],
+                res$Crisk[,"OAD" ,]), plot = TRUE,
+          xlim = c(0,10), xaxs = "i", yaxs = "i", las = 1,
+          xlab = "Time since DM diagnosis (years)", 
+          ylab = "Cumulative probability",
+           col = c("black","red","blue"))
> text(8, 0.3 + c(1,0,2)/25, 
+      c("Dead","Ins","OAD"),
+      col = c("black","red","blue"), adj = 0)
\end{Sinput}
\end{Schunk}
\insfig{crates}{0.9}{Cumulative risks for the three types of events,
  with 95\% bootstrap-based confidence intervals as shades.}

\section{Confidence intervals for stacked cumulative risks}

Unlike the single cumulative risks where we have a confidence
interval for each cumulative risk, when we want to show the stacked
probabilities we must deliver the confidence intervals for the
relevant sums, they are in the \texttt{Srisk} component of \texttt{res}.
\begin{Schunk}
\begin{Sinput}
> str(res$Crisk)
\end{Sinput}
\begin{Soutput}
 num [1:1001, 1:4, 1:3] 1 0.991 0.983 0.975 0.967 ...
 - attr(*, "dimnames")=List of 3
  ..$ time : chr [1:1001] "0" "1" "2" "3" ...
  ..$ cause: chr [1:4] "Surv" "OAD" "Ins" "Dead"
  ..$      : chr [1:3] "50%" "2.5%" "97.5%"
\end{Soutput}
\begin{Sinput}
> str(res$Srisk)
\end{Sinput}
\begin{Soutput}
 num [1:1001, 1:3, 1:3] 0 0.000696 0.001385 0.002065 0.002738 ...
 - attr(*, "dimnames")=List of 3
  ..$ time : chr [1:1001] "0" "1" "2" "3" ...
  ..$ cause: chr [1:3] "Dead" "Dead+Ins" "Dead+Ins+OAD"
  ..$      : chr [1:3] "50%" "2.5%" "97.5%"
\end{Soutput}
\end{Schunk}
But we start out by plotting the stacked probabilities using
\texttt{mat2pol} (\texttt{mat}rix to \texttt{pol}ygon), the input
required is the single components from the \texttt{Crisk}
component. Then we can add the confidence intervals
\begin{Schunk}
\begin{Sinput}
> zz <- mat2pol(res$Crisk[,c("Dead","Ins","OAD","Surv"),1],
+               x = as.numeric(dimnames(res$Crisk)[[1]])/100,
+            xlim = c(0,10), xaxs = "i", yaxs = "i", las = 1,
+            xlab = "Time since DM diagnosis (years)", 
+            ylab = "Probability",
+             col =  c("black","red","blue","forestgreen") )
> mm <- t(apply(zz, 1, mid<-function(x) x[-1] - diff(x) / 2))
> text( 9, mm[900,], c("Dead","Ins","OAD","DM"), col = "white" )
> matshade(as.numeric(dimnames(res$Srisk)[[1]])/100,
+          cbind(res$Srisk[,1,],
+                res$Srisk[,2,],
+                res$Srisk[,3,]),
+          col = 'transparent', col.shade = "white", alpha = 0.3)
\end{Sinput}
\end{Schunk}
\insfig{stack-ci}{1.0}{Probabilities of being in the 4 different
  states as a function of time since diagnosis. Note that \texttt{OAD}
  means that OAD was initiated first, and similarly for
  \texttt{Ins}. We are not concerned about what occurs \emph{after}
  these events.  \texttt{Dead} means dead without being on any
  drug.\newline The white shadings around the borders between coloured
  areas represent the 95\% confidence intervals for the (sum of)
  probabilities.}

\section{Sojourn times}

From the \texttt{Stime} component of the \texttt{res} we can derive
the estimated time spent in each state during the first, say, 5 or 10 years:
\begin{Schunk}
\begin{Sinput}
> str(res$Stime)
\end{Sinput}
\begin{Soutput}
 num [1:1000, 1:4, 1:3] 0.00996 0.01983 0.02962 0.03933 0.04896 ...
 - attr(*, "dimnames")=List of 3
  ..$      : chr [1:1000] "1" "2" "3" "4" ...
  ..$ cause: chr [1:4] "Surv" "OAD" "Ins" "Dead"
  ..$      : chr [1:3] "50%" "2.5%" "97.5%"
\end{Soutput}
\end{Schunk}
We extract the 5 and 10 years components:
\begin{Schunk}
\begin{Sinput}
> s510 <- res$Stime[1:2*500,,]
> dimnames(s510)[[1]] <- c(" 5 yr","10 yr")
> round(ftable(s510, row.vars=1:2), 2)
\end{Sinput}
\begin{Soutput}
              50% 2.5% 97.5%
      cause                 
 5 yr Surv   2.77 2.72  2.82
      OAD    1.44 1.40  1.49
      Ins    0.40 0.37  0.43
      Dead   0.39 0.36  0.42
10 yr Surv   4.31 4.22  4.41
      OAD    3.64 3.54  3.75
      Ins    0.84 0.78  0.90
      Dead   1.20 1.14  1.27
\end{Soutput}
\end{Schunk}
So we see that the expected life lived without pharmaceutical treatment
during the first 10 years after DM diagnosis is 4.31 years with a 95\%
CI of (4.21;4.42), and during the first 5 years 2.77 (2.72;2.82).

The quantity \texttt{OAD} is the years lived without medication that
has been terminated by OAD inception, and similarly for \texttt{Ins}
and \texttt{Dead}.

\end{document}
