%\VignetteIndexEntry{Competing risks with Lexis, parametric rates and simulation based confidence intervals}

\documentclass[a4paper,dvipsnames,twoside,12pt]{report}
\newcommand{\Title}{Competing risks with\\ \texttt{Lexis}, parametric
  rates and\\ simulation based confidence intervals}
\newcommand{\Tit}{CmpRskParSim}
\newcommand{\Version}{Version 6}
\newcommand{\Dates}{November 2024}
\newcommand{\Where}{SDCC}
\newcommand{\Homepage}{\url{http://bendixcarstensen.com/}}
\newcommand{\Faculty}{\begin{tabular}{rl}
Bendix Carstensen
  & Steno Diabetes Center Copenhagen, Herlev, Denmark\\
  & {\small \& Department of Biostatistics,
               University of Copenhagen} \\
  & \href{mailto:b@bxc.dk}{\tt b@bxc.dk} \\
  & \url{http://BendixCarstensen.com} \\[1em]
                      \end{tabular}}

%----------------------------------------------------------------------
% Packages
%\usepackage[inline]{showlabels}
%\usepackage[latin1]{inputenc}
\usepackage[utf8]{inputenc}
\usepackage[T1]{fontenc}
\usepackage[english]{babel}
\usepackage[font=it,labelfont=normalfont]{caption}
\usepackage[colorlinks,urlcolor=blue,linkcolor=red,citecolor=Maroon]{hyperref}
\usepackage[dvipsnames]{xcolor}
\usepackage[super]{nth}
% \usepackage[retainorgcmds]{IEEEtrantools}
\usepackage[noae]{Sweave}
\usepackage[ae,hyper]{Rd}
% \usepackage[noae]{C:/util/R/R-4.2.0/share/texmf/tex/latex/Sweave}
% \usepackage{C:/util/R/R-4.2.0/share/texmf/tex/latex/Rd}
\usepackage{makeidx,floatflt,amsmath,amsfonts,amsbsy,enumitem,dcolumn,needspace}
\usepackage{ifthen,calc,eso-pic,everyshi}
\usepackage{booktabs,longtable,rotating,graphicx,subfig}
\usepackage{pdfpages,verbatim,fancyhdr,datetime,afterpage}
\usepackage[abspath]{currfile}
% \usepackage{times}
\renewcommand{\textfraction}{0.0}
\renewcommand{\topfraction}{1.0}
\renewcommand{\bottomfraction}{1.0}
\renewcommand{\floatpagefraction}{0.9}
\definecolor{blaa}{RGB}{99,99,255}
\DeclareGraphicsExtensions{.png,.pdf,.jpg}
% Make the Sweave output nicer (slightly mor compact)
\DefineVerbatimEnvironment{Sinput}{Verbatim}{fontsize=\small,fontshape=sl,formatcom=\color{BlueViolet}}
\DefineVerbatimEnvironment{Soutput}{Verbatim}{fontsize=\small,formatcom=\color{Sepia},xleftmargin=0em}
\DefineVerbatimEnvironment{Scode}{Verbatim}{fontsize=\small}
\fvset{listparameters={\setlength{\topsep}{-0.1ex}}}
\renewenvironment{Schunk}%
{\renewcommand{\baselinestretch}{0.87} \vspace{\topsep}}%
{\renewcommand{\baselinestretch}{1.00} \vspace{\topsep}}
% \renewenvironment{knitrout}
% {\renewcommand{\baselinestretch}{0.87}}
% {\renewcommand{\baselinestretch}{1.00}}
% This is a file of useful extra commands snatched from
% Michael Hills, David Clayton, Bendix Carstensen & Esa Laara.
%

% Commands to draw observation lines on follow-up diagrams
%
% Horizontal lines
%
\providecommand{\hfail}[1]{\begin{picture}(250,5)
      \put(0,0){\line(1,0){#1}}
      \put(#1,0){\circle*{5}}
   \end{picture}}

\providecommand{\hcens}[1]{\begin{picture}(250,5)
      \put(0,0){\line(1,0){#1}}
      \put(#1,0){\line(0,1){2.5}}
      \put(#1,0){\line(0,-1){2.5}}
   \end{picture}}

%
% Diagonals for Lexis diagrams
%
\providecommand{\dfail}[1]{\begin{picture}(250,250)
      \put(0,0){\line(1,1){#1}}
      \put(#1,#1){\circle*{5}}
   \end{picture}}

\providecommand{\dcens}[1]{\begin{picture}(250,250)
      \put(0,0){\line(1,1){#1}}
%      \put(#1,#1){\line(0,1){2.5}}
%      \put(#1,#1){\line(0,-1){2.5}}
% BxC Changed this to an open circle instead of a line
      \put(#1,#1){\circle{5}}
   \end{picture}}

%
% Horizontal range diagrams
%
\providecommand{\hrange}[1]{\begin{picture}(200,5)
     \put(0,0){\circle*{5}}
     \put(0,0){\line(1,0){#1}}
     \put(0,0){\line(-1,0){#1}}
   \end{picture}}

%
% Tree drawing
%
\providecommand{\Tree}[3]{\setlength{\unitlength}{#1\unitlength}\begin{picture}(0,0)
   \put(0,0){\line(3, 2){1}}
   \put(0,0){\line(3,-2){1}}
   \put(0.81, 0.54){\makebox(0,0)[br]{\footnotesize #2\ }}
   \put(0.81,-0.54){\makebox(0,0)[tr]{\footnotesize #3\ }}
\end{picture}}

\providecommand{\Wtree}[3]{\setlength{\unitlength}{#1\unitlength}\begin{picture}(0,0)
   \put(0,0){\line(1, 1){1}}
   \put(0,0){\line(1,-1){1}}
   \put(0.8,0.8){\makebox(0,0)[br]{\footnotesize #2\ }}
   \put(0.8,-0.8){\makebox(0,0)[tr]{\footnotesize #3\ }}
\end{picture}}

\providecommand{\Ntree}[3]{\setlength{\unitlength}{#1\unitlength}\begin{picture}(0,0)
   \put(0,0){\line(2, 1){1}}
   \put(0,0){\line(2,-1){1}}
   \put(0.8,0.4){\makebox(0,0)[br]{\footnotesize #2\ }}
   \put(0.8,-0.4){\makebox(0,0)[tr]{\footnotesize #3\ }}
\end{picture}}

\providecommand{\Nutree}[3]{\setlength{\unitlength}{#1\unitlength}\begin{picture}(0,0)
   \put(0,0){\line(2, 1){#1}}
   \put(0,0){\line(2,-1){#1}}
   \put(0.8,0.4){\makebox(0,0)[br]{#2\ }}
   \put(0.8,-0.4){\makebox(0,0)[tr]{#3\ }}
\end{picture}}

%
% Tree drawing
%
\providecommand{\tree}[3]{\setlength{\unitlength}{#1}\begin{picture}(0,0)
   \put(0,0){\line(3,2){1}}
   \put(0,0){\line(3,-2){1}}
   \put(0.81,0.54){\makebox(0,0)[br]{\footnotesize #2\ }}
   \put(0.81,-0.54){\makebox(0,0)[tr]{\footnotesize #3\ }}
\end{picture}}

\providecommand{\wtree}[3]{\setlength{\unitlength}{#1}\begin{picture}(0,0)
   \put(0,0){\line(1,1){1}}
   \put(0,0){\line(1,-1){1}}
   \put(0.8,0.8){\makebox(0,0)[br]{\footnotesize #2\ }}
   \put(0.8,-0.8){\makebox(0,0)[tr]{\footnotesize #3\ }}
\end{picture}}

\providecommand{\ntree}[3]{\setlength{\unitlength}{#1}\begin{picture}(0,0)
   \put(0,0){\line(2,1){1}}
   \put(0,0){\line(2,-1){1}}
   \put(0.8,0.4){\makebox(0,0)[br]{\footnotesize #2\ }}
   \put(0.8,-0.4){\makebox(0,0)[tr]{\footnotesize #3\ }}
\end{picture}}

\providecommand{\nutree}[3]{\begin{picture}(0,0)
   \put(0,0){\line(2,1){#1}}
   \put(0,0){\line(2,-1){#1}}
   \put(0.8,0.4){\makebox(0,0)[br]{#2\ }}
   \put(0.8,-0.4){\makebox(0,0)[tr]{#3\ }}
\end{picture}}

%
% Other commands
%
\providecommand{\ip}[2]{\langle #1 \vert #2 \rangle} 
\providecommand{\I}{\text{\rm gI}}
\providecommand{\prob}[0]{\text{\rm Pr}}
\providecommand{\nhy}[0]{_{\oslash}}
\providecommand{\true}[0]{_{\text{\rm \tiny T}}}
\providecommand{\hyp}[0]{_{\text{\rm \tiny H}}}
% \providecommand{\mpydiv}[0]{\stackrel{\textstyle \times}{\div}}
% Changed to slightly smaller symbols
\providecommand{\mpydiv}[0]{\stackrel{\scriptstyle\times}{\scriptstyle\div}}
\providecommand{\mie}[1]{{\it #1}}
\providecommand{\mycircle}[0]{\circle*{5}}
\providecommand{\smcircle}[0]{\circle*{1}}
\providecommand{\corner}[0]{_{\text{\rm \tiny C}}}
\providecommand{\ind}[0]{\hspace{10pt}}
\providecommand{\gap}[0]{\\[5pt]}
\renewcommand{\S}[0]{section~}
\providecommand{\blank}[0]{$\;\,$}
\providecommand{\vone}{\vspace{1cm}}
\providecommand{\ljust}[1]{\multicolumn{1}{l}{#1}}
\providecommand{\cjust}[1]{\multicolumn{1}{c}{#1}}
\providecommand{\transpose}{^{\text{\sf T}}}
\providecommand{\histog}[5]{\rule{1mm}{#1mm}\,\rule{1mm}{#2mm}\,\rule{1mm}{#3mm}\,\rule{1mm}{#4mm}\,\rule{1mm}{#5mm}}
\providecommand{\pmiss}{P_{\mbox{\tiny miss}}}

% Below is BxCs commands inserted

% Only works with hyperref package:
\newcommand{\mailto}[1]{\href{mailto:#1}{\tt #1}}

\providecommand{\bc}{\begin{center}}
\providecommand{\ec}{\end{center}}
\providecommand{\bd}{\begin{description}}
\providecommand{\ed}{\end{description}}
\providecommand{\bi}{\begin{itemize}}
\providecommand{\ei}{\end{itemize}}
\providecommand{\bn}{\begin{equation}}
\providecommand{\en}{\end{equation}}
\providecommand{\be}{\begin{enumerate}}
\providecommand{\ee}{\end{enumerate}}
\providecommand{\bes}{\begin{eqnarray*}}
\providecommand{\ees}{\end{eqnarray*}}

\DeclareMathOperator{\Pp}{P}
\DeclareMathOperator{\pp}{p}
% \providecommand{\p}{{\mathrm p}}
\providecommand{\e}{{\mathrm e}}
\providecommand{\D}{{\mathrm D}}
\providecommand{\dif}{{\,\mathrm d}}
\providecommand{\pmat}[1]{\Pp\!\left\{#1\right\}}
\providecommand{\ptxt}[1]{\Pp\!\left\{\text{#1}\right\}}
\providecommand{\E}{\text{\rm E}}
\providecommand{\V}{\text{\rm V}}
\providecommand{\BLUP}{\text{\rm BLUP}}
\providecommand{\se}{\text{\rm s.e.}}
\providecommand{\sem}{\text{\rm s.e.m.}}
\providecommand{\std}{\text{\rm std}}
\providecommand{\sd}{\text{\rm s.d.}}
\providecommand{\Var}{\text{\rm var}}
\providecommand{\VAR}{\text{\rm var}}
\providecommand{\var}{\text{\rm var}}
\providecommand{\cov}{\text{\rm cov}}
\providecommand{\corr}{\text{\rm corr}}
\providecommand{\mean}{\text{\rm mean}}
\providecommand{\CV}{\text{\rm CV}}
\providecommand{\median}{\text{\rm median}}
\providecommand{\cv}{\text{\rm c.v.}}
\providecommand{\erf}{\text{\rm erf}}
\providecommand{\ef}{\text{\rm ef}}
\providecommand{\SSD}{\text{\rm SSD}}
\providecommand{\SPD}{\text{\rm SPD}}
\providecommand{\odds}{\text{\rm odds}}
\providecommand{\bin}{\text{\rm binom}}
\providecommand{\half}{\frac{1}{2}}
% \providecommand{\td}[0]{\stackrel{\textstyle \times}{\div}}
% Changed to slightly smaller symbols
\providecommand{\td}[0]{\stackrel{\scriptstyle \times}{\scriptstyle \div}}
\providecommand{\dt}[0]{\stackrel{\scriptstyle \div}{\scriptstyle \times}}
\providecommand{\diag}{\text{\rm diag}}
\providecommand{\det}{\text{\rm det}}
\providecommand{\dim}{\text{\rm dim}}
\providecommand{\spcol}{\text{\rm span}}
\providecommand{\logit}{\text{\rm logit}}
% \providecommand{\link}{\text{\rm link}}
\providecommand{\spn}{\text{\rm span}}
\providecommand{\CI}{\text{\rm CI}}
\providecommand{\IP}{\text{\rm IP}}
\providecommand{\OR}{\text{\rm OR}}
\providecommand{\RR}{\text{\rm RR}}
\providecommand{\ER}{\text{\rm ER}}
\providecommand{\EM}{\text{\rm EM}}
\providecommand{\EF}{\text{\rm EF}}
\providecommand{\RD}{\text{\rm RD}}
\providecommand{\AC}{\text{\rm AC}}
\providecommand{\AF}{\text{\rm AF}}
\providecommand{\PAF}{\text{\rm PAF}}
\providecommand{\AR}{\text{\rm AR}}
\providecommand{\CR}{\text{\rm CR}}
\providecommand{\PAR}{\text{\rm PAR}}
\providecommand{\EL}{\text{\rm EL}}
\providecommand{\ERL}{\text{\rm ERL}}
\providecommand{\YLL}{\text{\rm YLL}}
\providecommand{\SD}{\text{\rm SD}}
\providecommand{\SE}{\text{\rm SE}}
\providecommand{\SEM}{\text{\rm SEM}}
\providecommand{\SR}{\text{\rm SR}}
\providecommand{\SMR}{\text{\rm SMR}}
\providecommand{\RSR}{\text{\rm RSR}}
\providecommand{\CMF}{\text{\rm CMF}}
\providecommand{\pvp}{\text{\rm PV$+$}}
\providecommand{\pvn}{\text{\rm PV$-$}}
\providecommand{\R}{{\textsf{\textbf{R}}}}
\providecommand{\sas}{\textsl{\textbf{SAS}}}
\providecommand{\SAS}{\textsl{\textbf{SAS}}}
%\providecommand{\gap}[0]{\\[5pt]}
%\providecommand{\blank}[0]{$\;\,$}
% Conditional independence sign from Philip Dawid
\providecommand{\cip}{\mbox{$\perp\!\!\!\perp$}}

%%% Commands to comment out parts of the text
\providecommand{\GLEM}[1]{}
\providecommand{\FORGETIT}[1]{}
\providecommand{\OMIT}[1]{}

%%% Insert output from program in small text
%%% (requires package verbatim)
\providecommand{\insoutsmall}[1]{
% \small
 \footnotesize
 \renewcommand{\baselinestretch}{0.8}
 \verbatiminput{#1}
 \renewcommand{\baselinestretch}{1.0}
 \normalsize
}
\providecommand{\insout}[1]{
 \scriptsize
 \renewcommand{\baselinestretch}{0.8}
 \verbatiminput{#1}
 \renewcommand{\baselinestretch}{1.0}
 \normalsize
}
\providecommand{\insouttiny}[1]{
\tiny
\renewcommand{\baselinestretch}{0.8}
\verbatiminput{#1}
\renewcommand{\baselinestretch}{1.0}
\normalsize
}

% From Esa:
\providecommand{\T}{\text{\rm \small{T}}}
\providecommand{\id}{\text{\rm id}}
\providecommand{\Dev}{\text{\rm Dev}}
\providecommand{\Bin}{\text{\rm Bin}}
\providecommand{\probit}{\text{\rm probit}}
\providecommand{\cloglog}{\text{\rm cloglog}}

% Special commands to include output from R, Bugs and Stata

\providecommand{\Rin}[2]{
\subsection{\texttt{#1.R}}
#2

\insout{./R/#1.Rout}

}

\providecommand{\Statain}[2]{
\subsection{\texttt{#1.do}}
#2

\insout{./do/#1.log}

}

\providecommand{\Bugsin}[2]{
\subsection{\texttt{#1.bug}}
#2

\insout{./bugs/#1.bug}

}

\newlength{\wdth}
\providecommand{\fxbl}[1]{\settowidth{\wdth}{#1} \makebox[\wdth]{}}

%%% Local Variables:
%%% mode: latex
%%% TeX-master: t
%%% End:


%----------------------------------------------------------------------
% Set up layout of pages
\oddsidemargin 1mm
\evensidemargin 1mm
\topmargin -10mm
\headheight 8mm
\headsep 5mm
\textheight 240mm
\textwidth 165mm
%\footheight 5mm
\footskip 15mm
\renewcommand{\topfraction}{0.9}
\renewcommand{\bottomfraction}{0.9}
\renewcommand{\textfraction}{0.1}
\renewcommand{\floatpagefraction}{0.9}
\renewcommand{\headrulewidth}{0.1pt}
\setcounter{secnumdepth}{2}
\setcounter{tocdepth}{3}

%----------------------------------------------------------------------
% How to insert a figure in a .rnw file
\newcommand{\rwpre}{./graph/gr}
\newcommand{\insfig}[3]{
\begin{figure}[h]
  \centering
  \includegraphics[width=#2\textwidth]{\rwpre-#1}
% \caption{#3}
  \caption{#3\hfill\mbox{\footnotesize \textrm{\tt \rwpre-#1}}}
  \label{fig:#1}
% \afterpage{\clearpage}
\end{figure}}
\newcommand{\linput}[1]{
% \clearpage 
\afterpage{\hfill \ldots now input from \texttt{#1.tex}\\} 
\fancyfoot[OR,EL]{\footnotesize \texttt{#1.tex}} 
\input{#1}}

%----------------------------------------------------------------------
% Here is the document starting with the titlepage
\begin{document}

%----------------------------------------------------------------------
% The title page
\setcounter{page}{1}
\pagenumbering{roman}
\pagestyle{plain}
\thispagestyle{empty}
% \vspace*{0.05\textheight}
\flushright
% The blank below here is necessary in order not to muck up the
% linespacing in title if it has more than 2 lines
{\Huge \bfseries \Title

}\ \\[-1.5ex]
\noindent\textcolor{blaa}{\rule[-1ex]{\textwidth}{5pt}}\\[2.5ex]
\large
\Where \\
\Dates \\
\Homepage \\
\Version \\[1em]
\normalsize
Compiled \today,\ \currenttime\\
from: \texttt{\currfileabspath}\\[1em]
% \input{wordcount}
\normalsize
\vfill
\Faculty
% End of titlepage
\newpage

%----------------------------------------------------------------------
% Table of contents
\tableofcontents
% \listoftables
% \listoffigures
\clearpage
% \begingroup
% \let\clearpage\relax
% \listoftables
% \listoffigures
% \endgroup

%----------------------------------------------------------------------
% General text layout
\raggedright
\parindent 1em
\parskip 0ex
\cleardoublepage

%----------------------------------------------------------------------
% General page style
\pagenumbering{arabic}
\setcounter{page}{1}
\pagestyle{fancy}
\renewcommand{\chaptermark}[1]{\markboth{\textsl{#1}}{}}
\renewcommand{\sectionmark}[1]{\markright{\thesection\ \textsl{#1}}{}}
\fancyhead[EL]{\bf \thepage \quad \rm \rightmark}
\fancyhead[ER]{\rm \Tit}
\fancyhead[OL]{\rm \leftmark}
\fancyhead[OR]{\rm \rightmark \quad \bf \thepage}
\fancyfoot{}

\renewcommand{\rwpre}{./03crisk}

\chapter{Competing risks}

\section{Technicalities}
First we set some output and graphics parameters for convenience and
load the packages needed:
\begin{Schunk}
\begin{Sinput}
> options(width = 90,
+         show.signif.stars = FALSE,
+         SweaveHooks=list(fig = function()
+                          par(mar = c(3, 3, 1, 1),
+                              mgp = c(3, 1, 0) / 1.6,
+                              las = 1,
+                             lend = "butt",
+                              bty = "n")))
> library(Epi)
> library(popEpi)
> library(survival)
> clear()
\end{Sinput}
\end{Schunk}
% must be after clear() because 'anfang' is used at the end
\begin{Schunk}
\begin{Soutput}
     R   Epi   popEpi
 4.5.0  2.59   0.4.13
\end{Soutput}
\end{Schunk}

\section{Concepts}

The concept of competing risks is one where persons in a given state,
``alive'', say, are subject to a number of different causes of death,
``cause1'', ``cause2'' etc. Causes of death are required to be
exhaustive and mutually exclusive. That is, you will eventually die
from one of the designated causes, and you can only die from one.

The \emph{cause-specific} rate for cause $c$ is defined as:
\[
   \lambda_c(t) = \ptxt{death from cause $c$ in $(t,t+h]\,|\,$
                        alive at $t$} / h
\]
\ldots formally, the limit of this quantity as $h\rightarrow 0$.

The observed data will be a survival time and an exit status which is
either ``censored alive'' or one of the causes. In situations where
the causes are not causes of death but other events, it is implicit
that we only consider the first occurrence of an event from the state
``alive'', and ignore whatever occurs after that.

\subsection{Cause specific rates and likelihood}

The likelihood for observations from a competing risk scenario is a
function of the cause-specific transition rates, and is a
\emph{product} of the likelihoods that would emerge if we considered
each cause as being the only possible event. Thus analysis is in
principle straight forward: estimate a model for each of the
cause-specific rates; these will together form a complete model for
the competing risks problem.

If the cause-specific rates are all we want to assess then we are done.

\subsection{Survival and cumulative risks}

In addition to the rates we might however also be interested in the
\emph{survival} probability and the \emph{cumulative risks} of each
cause of death.

The survival is the probability of still being alive at a given time
after entry---a function of time since entry. The cumulative risk of
dying from cause $c$ is the probability of having died from cause $c$
as a function of time since entry.

This means that a time of entry is required for the calculations these
quantities.

\subsection{Sojourn times}

The \emph{sojourn time} for cause $c$ is the time spent in the ``cause
$c$'' state before a given time, $t$, say. This is also called the
expected lifetime lost to cause $c$, \emph{truncated} at the time
$t$. For the state ``alive'' it will be the expected time lived before
$t$. This is also called the restricted mean survival time, RMST.

\subsection{The time scale}

The cause specific rates will be functions of covariates, notably
a time scale, be that age or time since entry to the study or even
calendar time. But the cumulative risks are probabilities that refer
to time \emph{since} some origin. Thus cumulative risks (and survival)
are only meaningful in relation to a time that begins at 0. Though not
a formal mathematical requirement this implies that we should have
data starting at time 0.

If we were to use age as timescale for cumulative risk, we would want
data available since birth; if we only had observations where most
people entered between 20 and 40 years of age, we could mathematically
compute cumulative risk to some age, but it would be nonsense. Instead we
would compute the cumulative risk \emph{given} that a person attained
age 40, say. In that case the time scale would be $\text{age} - 40$.

\section{Rates and cumulative risks}

Each of the cumulative risks is a function of the survival function which
in turn depends on \emph{all} rates. Specifically, if the
cause-specific rates are $\lambda_c(t),\,c=1,2,\ldots$, then the
survival function (probability of being alive at time $t$) is:
\begin{equation}
     S(t) = \exp\left( \!-\!\int_0^t \sum_c \lambda_c(s) \dif s \right)
     =  \exp\left(\!-\! \sum_c \Lambda_c(t) \right)
          \label{eq:Sv}
\end{equation}
The quantities $\Lambda_c(t) = \int_0^t \lambda_c(s) \dif s$ are
called cumulative \emph{rates} (probabilists call them integrated
intensities), although they are not rates. Cumulative rates are
dimensionless, but they have no probability interpretation of any
kind.

The cumulative \emph{risk}, the probability of dying from cause $k$,
say, before time $t$, $R_k(t)$ is:
\begin{equation}
   R_k(t) = \int_{u=0}^{u=t} \!\! \lambda_k(u) S(u) \dif u
           = \int_{u=0}^{u=t} \!\! \lambda_k(u) \exp\left(\!-\!
            \sum_j \Lambda_j(u) \right) \! \dif u
          \label{eq:R}
\end{equation}
The rationale is that $\lambda_k(u) \dif u$ is the probability of
death from cause $k$ in the small interval $(u, u + \dif u)$,
\emph{conditional} on being alive at time $u$. If this is multiplied
with the probability of being alive at $u$, $S(u)$, Bayes rule tells
us that we get the \emph{marginal probability} of death from cause $k$
in the small interval $(u, u + \dif u)$. This function of $u$ is the
argument in the integral; so integration from $u=0$ to $u=t$, will
give the probability of death from cause $c$ anywhere in $(0,t)$---the
cumulative risk of cause $k$ at $t$.

Parametric models for the cause-specific rates can produce estimated
transition rates $\lambda_c$ at closely spaced intervals, and the
cumulative risks can then be estimated from these by simple numerical
integration; this is illustrated in the next chapter.

Note that at any one time every person is either alive or dead from one
of the causes, so the sum of the survival and the cumulative risks is
always 1:
\[
   1 = S(t) + R_1(t) + R_2(t) + \cdots, \quad \forall t
\]

\subsection{Confidence intervals by simulation}

But even if we from the modeling of the $\lambda_c$s may have standard
errors of $\log\big(\lambda_c(t)\big)$, the standard errors of the
$R_c$s will be analytically intractable from these.

In practice, the only viable way to get confidence intervals for the
cumulative risks, $R_c$, is therefore by calculation of a set of rates
$\lambda_c(t)$ by sampling from the posterior distribution of the
parameters in the models for $\log\big(\lambda_c(t)\big)$, and then
compute the integrals numerically for each simulated sample, according
to formulae \ref{eq:Sv} and \ref{eq:R}. This will produce a so-called
parametric bootstrap sample of the cumulative risks from which we can
derive confidence intervals

The simulation approach also allows calculation of confidence
intervals for sums of the cumulative risks, $R_1(t)+R_2(t)$, for
example, which will be needed if we want to show stacked cumulative
risks.

Finally, it will also allow calculation of standard errors of sojourn
times in each of the states ``alive'' and ``cause1'',
``cause2'',\ldots. While the latter two may not be of direct interest,
then \emph{differences} between such sojourn times between different
groups can be interpreted as years of life lost to each cause between
groups.

\subsection{Subdistribution hazard}

A common concept seen in competing risks is the subdistribution
hazard, and proportional hazards models for this (the Fine-Gray model).

Suppose for a moment we only consider all-cause mortality,
$\lambda(t)$. Then the cumulative risk of death is:
\[
    R(t) = 1 - S(t) = 1 - \exp\left( \!-\!\int_0^t \lambda(s) \dif s \right)
\]
Solving this for $\lambda$ will yield:
\[
  \lambda(t) = -\frac{\dif\log\big(1 - R(t)\big)}{\dif t}
\]
In the case of multiple causes of death we can define the
\emph{subdistribution hazard} for cause $c$ by using the same
transformation of the cumulative risk for cause $c$:
\[
  \tilde\lambda_c(t) = -\frac{\dif\log\big(1 - R_c(t)\big)}{\dif t}
\]
or, for the cause-specific risk:
\[
    R_c(t) = 1 - \exp\left( \!-\!\int_0^t \tilde\lambda_c(s) \dif s \right)
\]
Thus, the subdistribution hazard $\tilde\lambda_c(t)$ is a function
which, when subjected to the (hazard$\rightarrow$risk) function from
the all-cause mortality case, yields the cause-specific risk.

The Fine-Gray model is a model for the subdistribution hazard
$\tilde\lambda_c$. It is only a model for \emph{one} cause-specific
hazard. Of course it can be applied to all available causes in turn,
but the sum of the cumulative risks derived from the models may exceed
1\ldots

Unlike a cause-specific hazard, which can depend on multiple time
scales, the subdistribution hazard can only depend on one, since it
requires an origin---just like cumulative risks.

But the interpretation of a subdistribution hazard is difficult.  I
have yet to see one that goes beyond the mathematical formalism
above. Therefore the subdistribution is not included in this vignette.

\chapter{Example data}

\section{A \texttt{Lexis} object}

As an illustrative data example we use the (fake) diabetes register
data; we set up the Lexis object, an then cut the follow-up time at
dates of \texttt{OAD} and \texttt{Ins} using \texttt{mcutLexis}:
\begin{Schunk}
\begin{Sinput}
> data(DMlate)
> Ldm <- Lexis(entry = list(per = dodm,
+                           age = dodm-dobth,
+                           tfd = 0),
+               exit = list(per = dox),
+        exit.status = factor(!is.na(dodth), labels = c("DM", "Dead")),
+               data = DMlate)
\end{Sinput}
\begin{Soutput}
NOTE: entry.status has been set to "DM" for all.
NOTE: Dropping  4  rows with duration of follow up < tol
\end{Soutput}
\begin{Sinput}
> summary(Ldm, t = T)
\end{Sinput}
\begin{Soutput}
Transitions:
     To
From   DM Dead  Records:  Events: Risk time:  Persons:
  DM 7497 2499      9996     2499   54273.27      9996

Timescales:
per age tfd 
 ""  ""  "" 
\end{Soutput}
\begin{Sinput}
> set.seed(1952)
> Mdm <- mcutLexis(Ldm,
+                   wh = c('dooad','doins'),
+           new.states = c('OAD','Ins'),
+           seq.states = FALSE,
+                 ties = TRUE)
\end{Sinput}
\begin{Soutput}
NOTE: Precursor states set to DM 
NOTE: 15 records with tied events times resolved (adding 0.01 random uniform),
      so results are only reproducible if the random number seed was set.
\end{Soutput}
\begin{Sinput}
> summary(Mdm)
\end{Sinput}
\begin{Soutput}
Transitions:
     To
From        DM Dead  OAD  Ins Ins+OAD  Records:  Events: Risk time:  Persons:
  DM      2830 1056 2957  689       0      7532     4702   22920.38      7532
  OAD        0  992 3327    0    1005      5324     1997   22965.24      5324
  Ins        0  152    0  462     172       786      324    3883.06       786
  Ins+OAD    0  299    0    0     878      1177      299    4504.58      1177
  Sum     2830 2499 6284 1151    2055     14819     7322   54273.27      9996
\end{Soutput}
\end{Schunk}
before drug inception) in intervals of 1/12 year, creating a
\texttt{Lexis} object for a competing risks situation with three
possible event types:
\begin{Schunk}
\begin{Sinput}
> Sdm <- splitLexis(factorize(subset(Mdm,
+                                    lex.Cst == "DM")),
+                   time.scale = "tfd",
+                       breaks = seq(0, 20, 1/12))
> summary(Sdm)
\end{Sinput}
\begin{Soutput}
Transitions:
     To
From     DM Dead  OAD Ins  Records:  Events: Risk time:  Persons:
  DM 274263 1056 2957 689    278965     4702   22920.38      7532
\end{Soutput}
\end{Schunk}
We can illustrate the follow-up in the full data frame and in the
restricted data frame
\begin{Schunk}
\begin{Sinput}
> boxes(Mdm, boxpos = list(x = c(15, 50, 15, 85, 85),
+                          y = c(85, 50, 15, 85, 15)),
+           scale.R = 100,
+           show.BE = TRUE)
\end{Sinput}
\end{Schunk}
\insfig{boxes5}{0.8}{The transitions in the multistate model, where
    follow-up is extended also after beginning of first drug
    exposure. Rates in brackets are per 100 PY.}%
\begin{Schunk}
\begin{Sinput}
> boxes(Relevel(Sdm, c(1, 4, 2, 3)),
+       boxpos  = list(x = c(15, 85, 75, 15),
+                      y = c(85, 85, 30, 15)),
+       scale.R = 100,
+       show.BE = TRUE )
\end{Sinput}
\end{Schunk}
\insfig{boxes4}{0.8}{The transitions in the competing risks model,
  where follow-up is stopped at first drug exposure. By that token only the
  \texttt{DM} state has person-years; a characteristic of a competing risks
  situation.}

\section{Models for rates}

Now that we have set up a dataset with three competing events, we can
model the cause-specific rates separately by time from diagnosis as
the only underlying time scale.

This is done by Poisson-regression on the time-split data set; since
the dataset is in \texttt{Lexis} format we can use the convenience
wrapper \texttt{gamLexis} to model rates as smooth functions of time
(\texttt{tfd}).  Note that we only need to specify the \texttt{to=}
argument because there is only one possible \texttt{from} for each
\texttt{to} (incidentally the same for all \texttt{to} states, namely
\texttt{DM}):
\begin{Schunk}
\begin{Sinput}
> mD <- gamLexis(Sdm, ~ s(tfd, k = 5), to = 'Dead')
\end{Sinput}
\begin{Soutput}
mgcv::gam Poisson analysis of Lexis object Sdm with log link:
Rates for the transition:
DM->Dead
\end{Soutput}
\begin{Sinput}
> mO <- gamLexis(Sdm, ~ s(tfd, k = 5), to = 'OAD' )
\end{Sinput}
\begin{Soutput}
mgcv::gam Poisson analysis of Lexis object Sdm with log link:
Rates for the transition:
DM->OAD
\end{Soutput}
\begin{Sinput}
> mI <- gamLexis(Sdm, ~ s(tfd, k = 5), to = 'Ins' )
\end{Sinput}
\begin{Soutput}
mgcv::gam Poisson analysis of Lexis object Sdm with log link:
Rates for the transition:
DM->Ins
\end{Soutput}
\end{Schunk}
We see that \texttt{gamLexis} (just like \texttt{glmLexis} would)
tells us what transition rates are modeled.

With these models fitted we can compute the rates, and from these
cumulative rates and the cumulative risks and sojourn times in states
using the usual formulae.

First we compute the rates in intervals of length 1/20 years. Note
that these prediction points are unrelated to the follow-up intervals
in which we split the observed data for analysis---they were 1 month
intervals (1/12 year), here we use 1/20 year (in real
life a smaller interval should be used, say 1/50 or 1/100 year):
\begin{Schunk}
\begin{Sinput}
> nd <- data.frame(tfd = seq(0, 10, 1/20))
> rownames(nd) <- nd$tfd
> str(nd)
\end{Sinput}
\begin{Soutput}
'data.frame':	201 obs. of  1 variable:
 $ tfd: num  0 0.05 0.1 0.15 0.2 0.25 0.3 0.35 0.4 0.45 ...
\end{Soutput}
\end{Schunk}
With this we can show the rates as a function of the time since entry
(diagnosis of diabetes):
\begin{Schunk}
\begin{Sinput}
> matshade(nd$tfd, cbind(ci.pred(mD, nd),
+                        ci.pred(mI, nd),
+                        ci.pred(mO, nd)) * 1000,
+          col = c("black", "red", "blue"), log = "y", lwd = 3, plot = TRUE,
+          xlab = "Time since DM diagnosis (years)",
+          ylab = "Rates per 1000 PY", ylim = c(0.05, 500), yaxt = "n")
> axis(side = 2, at = ll <- outer(c(1,2,5), -2:3, function(x,y) x * 10^y),
+                labels = formatC(ll, digits = 4), las = 1)
> axis(side = 2, at = outer(c(1.5,2:9), -2:3, function(x,y) x * 10^y),
+                labels = NA, tcl = -0.3)
> text(0, 0.5*0.6^c(1,2,0),
+      c("Dead","Ins","OAD"),
+      col = c("black","red","blue"), adj = 0, font = 2)
\end{Sinput}
\end{Schunk}
\insfig{rates}{0.8}{Estimated rates from the \textrm{\tt DM} state,
  estimates are from \textrm{\tt gam} models fitted to data split in 1
  month intervals (1/12 year, that is). Rates of \textrm{\tt OAD} is
  in the vicinity of 0.1/year, and mortality about half of this. Rates
  of insulin start among persons on no other drug are beginning high,
  then decreasing with a nadir at about 4 years and then increase to a
  peak at 8 years.}

Note that the graph in figure \ref{fig:rates} is not normally shown in
analyses of competing risks; the competing cause-specific \emph{rates}
are hardly ever shown. I suspect that this is frequently because they
are often modeled by a Cox model and so are buried in the model and
hard to get at.

Since we will be integrating the rates, it would also be relevant to show
the rates on a linear scale instead, de-emphasizing the very small
fluctuations of the \texttt{Ins} rates that are over-emphasized when
using a log-scale for the $y$-axis.
\begin{Schunk}
\begin{Sinput}
> matshade(nd$tfd, cbind(ci.pred(mD, nd),
+                        ci.pred(mI, nd),
+                        ci.pred(mO, nd)) * 1000,
+          col = c("black", "red", "blue"), lwd = 3, plot = TRUE,
+          xlab = "Time since DM diagnosis (years)",
+          ylab = "Rates per 1000 PY", ylim = c(0, 500), yaxs = "i")
> text(8, 500 - c(2, 3, 1) * 20,
+      c("Dead","Ins","OAD"),
+      col = c("black","red","blue"), adj = 0, font = 2)
\end{Sinput}
\end{Schunk}
\insfig{rates-l}{0.8}{Estimated rates from the \textrm{\tt DM} state,
  estimates are from \textrm{\tt gam} models fitted to data split in 1
  month intervals (1/12 year, that is). Rates of \textrm{\tt OAD} is
  in the vicinity of 0.1/year, and mortality about half of this. .}

\section{Cumulative rates and risks}

For the calculation of the cumulative rates and state probabilities,
we need just the estimated rates (without CIs). The formulae
\ref{eq:Sv} and \ref{eq:R} on page \pageref{eq:R} are transformed to
\R-code; starting with the rates, $\lambda_\text{D}$ as \texttt{lD}
etc:
\begin{Schunk}
\begin{Sinput}
> # utility function to compute midpoints between sucessive values in a vector
> mp <- function(x) x[-1] - diff(x) / 2
> #
> int <- 1 / 20
> # rates at midpoints of intervals
> lD <- mp(ci.pred(mD, nd)[, 1])
> lI <- mp(ci.pred(mI, nd)[, 1])
> lO <- mp(ci.pred(mO, nd)[, 1])
> #
> # cumulative rates and survival function at right border of the intervals
> LD <- cumsum(lD) * int
> LI <- cumsum(lI) * int
> LO <- cumsum(lO) * int
> # survival function, formula (1.1)
> Sv <- exp(- LD - LI - LO)
> #
> # when integrating to get the cumulative *risks* we use the average
> # of the survival function at the two endpoints
> # (adding 1 as the first), formula (1.2)
> Sv <- c(1, Sv)
> rD <- c(0, cumsum(lD * mp(Sv)) * int)
> rI <- c(0, cumsum(lI * mp(Sv)) * int)
> rO <- c(0, cumsum(lO * mp(Sv)) * int)
\end{Sinput}
\end{Schunk}
Now we have the cumulative risks for the three causes and the
survival, computed at the end of each of the intervals. At any time
point the sum of the 3 cumulative risks and the survival should be 1:
\begin{Schunk}
\begin{Sinput}
> summary(rD + rI + rO + Sv)
\end{Sinput}
\begin{Soutput}
   Min. 1st Qu.  Median    Mean 3rd Qu.    Max. 
      1       1       1       1       1       1 
\end{Soutput}
\begin{Sinput}
> oo <- options(digits = 20)
> cbind(summary(Sv + rD + rI + rO))
\end{Sinput}
\begin{Soutput}
                         [,1]
Min.    1.0000000000000000000
1st Qu. 1.0000252933615676465
Median  1.0000254405444475303
Mean    1.0000249140597872177
3rd Qu. 1.0000258488927373790
Max.    1.0000260207157363190
\end{Soutput}
\begin{Sinput}
> options(oo)
\end{Sinput}
\end{Schunk}
\ldots and bar a small rounding error, they are.

We can then plot the 3 cumulative risk functions stacked together
using \texttt{mat2pol} (\texttt{mat}rix to \texttt{pol}ygons):
\begin{Schunk}
\begin{Sinput}
> zz <- mat2pol(cbind(rD, rI, rO, Sv), x = nd$tfd, # $
+               xlim = c(0,10), xaxs = "i", yaxs = "i", las = 1,
+               xlab = "Time since DM diagnosis (years)",
+               ylab = "Probability",
+                col =  c("black","red","blue","forestgreen"))
> text(9, mp(zz["9", ]), c("Dead", "Ins", "OAD"," DM"), col = "white")
> box(col = "white", lwd = 3)
\end{Sinput}
\end{Schunk}
\insfig{stack}{0.9}{Probabilities of being in the 4 different states
  as a function of time since diagnosis. Note that \textrm{\tt OAD} means
  that OAD was initiated first, and similarly for \textrm{\tt Ins}. We are
  not concerned about what occurs after any these events.  \textrm{\tt Dead}
  means dead without initiating any of the two drugs.}

\section{Sojourn times}

The sojourn times in each of the states is just the area of each of
the coloured parts of figure \ref{fig:stack}. Since the $y$-dimension
of the plot is probability (dimensionless) and the $x$-axis has
dimension time, the computed areas will have dimension time.

Normally we will not report the sojourn times as functions of
(truncation) time, but only report them at a few select truncation
points, such as 5 or 10 years. Calculation of the 10 year sojourn
times would be straight-forward as integrals from 0 to 10---these
calculations rely on the predicted rates from \texttt{nd} being for
the first 10 years:
\begin{Schunk}
\begin{Sinput}
> Sj <- c(sjA = sum(Sv * int),
+         sjD = sum(rD * int),
+         sjI = sum(rI * int),
+         sjO = sum(rO * int))
> c(Sj, sum(Sj))
\end{Sinput}
\begin{Soutput}
       sjA        sjD        sjI        sjO            
 4.3486390  1.2071800  0.8396038  3.6548276 10.0502504 
\end{Soutput}
\end{Schunk}
We see that there is a some rounding error in the calculations; the
sum should really be exactly 10.

This was a demonstration on how to compute the rates, cumulative risks
and sojourn times. But no confidence intervals.

\chapter{Confidence intervals for cumulative risks}

Besides confidence intervals for each of the 4 cumulative risks,
we will also be interested in confidence intervals for \emph{sums} of
any subset of the cumulative risks, corresponding to the borders
between the colours in figure \ref{fig:stack}. If we only had two
competing risks (and hence three states) the latter would not be an
issue, because the sum of any two cumulative risks will be 1 minus the
cumulative risk of the remainder, so we could get away with the
confidence intervals for the single cumulative risks. This is the
reason we have chosen an example with 3 competing risks and not just
2; we then have 4 probabilities to sum in different order.

A short look at the formulae for cumulative risks will reveal that
analytic approximation to the standard error of these probabilities
(or some transform of them) is not really a viable way to
go. Particularly if we also want confidence intervals of sums of the
state probabilities as those shown in stacked plots.

So in practice, if we want confidence intervals not only for the state
probabilities, but also for any sum of subsets of them we would want a
large number of simulated copies of the cumulative risks, each copy
being of the same structure as the one we just extracted from the
models.

Confidence intervals for sojourn times (i.e. time spent) in each state
up to a given time, would come almost for free from the simulation
approach, by taking the relevant quantiles of the simulated quantities.

This means that we must devise a method to make a prediction not from
the estimated model, but where we instead of the model parameters use
a sample from the posterior distribution of the estimated parameters.
Here, the posterior distribution of the parameters will be taken to be
the multivariate normal distribution with mean equal to the vector of
parameter estimates and variance-covariance matrix equal to the
estimated variance-covariance matrix of the parameters.

Precisely this approach is implemented in \texttt{ci.lin} via the
\texttt{sample} argument; we can get a predicted value from a given
prediction data frame just as from \texttt{ci.pred}
resp. \texttt{ci.exp}; here is shown two different ways of getting
predicted values of the cause-specific rates:
\begin{Schunk}
\begin{Sinput}
> head(cbind(ci.pred(mI, nd),
+            ci.exp (mI, nd)))
\end{Sinput}
\begin{Soutput}
      Estimate      2.5%     97.5% exp(Est.)      2.5%     97.5%
0    0.3425874 0.3111484 0.3772030 0.3425874 0.3111484 0.3772030
0.05 0.2874359 0.2630855 0.3140401 0.2874359 0.2630855 0.3140401
0.1  0.2411669 0.2221042 0.2618657 0.2411669 0.2221042 0.2618657
0.15 0.2023535 0.1871526 0.2187890 0.2023535 0.1871526 0.2187890
0.2  0.1697982 0.1573575 0.1832226 0.1697982 0.1573575 0.1832226
0.25 0.1424958 0.1319999 0.1538263 0.1424958 0.1319999 0.1538263
\end{Soutput}
\end{Schunk}
Here is an illustration of the prediction with model based confidence
intervals for the rates of insulin start (model \texttt{mI}),
alongside predictions based on samples from the posterior distribution
of the parameters in the model:
\begin{Schunk}
\begin{Sinput}
> str(ci.lin(mI, nd, sample = 4))
\end{Sinput}
\begin{Soutput}
 num [1:201, 1:4] -1.01 -1.2 -1.39 -1.57 -1.76 ...
 - attr(*, "dimnames")=List of 2
  ..$ : chr [1:201] "0" "0.05" "0.1" "0.15" ...
  ..$ : NULL
\end{Soutput}
\begin{Sinput}
> head(cbind(ci.pred(mI, nd), exp(ci.lin(mI, nd, sample = 4))))
\end{Sinput}
\begin{Soutput}
      Estimate      2.5%     97.5%                                        
0    0.3425874 0.3111484 0.3772030 0.3542825 0.3503058 0.3431391 0.3784217
0.05 0.2874359 0.2630855 0.3140401 0.2941846 0.2903514 0.2894193 0.3123394
0.1  0.2411669 0.2221042 0.2618657 0.2442853 0.2406626 0.2441134 0.2578015
0.15 0.2023535 0.1871526 0.2187890 0.2028578 0.1994859 0.2059075 0.2127957
0.2  0.1697982 0.1573575 0.1832226 0.1684676 0.1653675 0.1736930 0.1756604
0.25 0.1424958 0.1319999 0.1538263 0.1399229 0.1371016 0.1465341 0.1450233
\end{Soutput}
\end{Schunk}
Note that we use \texttt{exp(ci.lin(...}---this is because the
\texttt{sample=} argument does not work with \texttt{ci.exp}.

The simulation (parametric bootstrapping) is taking place at the
parameter level and the transformation to survival and cumulative
risks is simply by a function applied to each simulated set of rates.

\section{Common parameters across cause-specific rates}

Note that we have implicitly been assuming that the transitions are
being modeled separately. If some transitions are modeled
jointly---for example assuming that the rates of \texttt{OAD} and
\texttt{Ins} are proportional as functions of time since entry, we are
in trouble, because we then need one sample from the posterior
generating two different predictions, one for each of the transitions
modeled together. Moreover the model will have to be a model fitted to
a \texttt{stack.Lexis} object, so a little more complicated to work
with.

A simple way to program this would be to reset the seed to the same
value before simulating with different values of \texttt{nd}, this is
what is intended to be implemented, but is not yet. This is mainly the
complication of having different prediction frames for different risks
in this case.

However, this is not a very urgent need, since the situation where you
want common parameters for different rates out of a common state is
quite rare. It would for example be quite odd to assume the the M/W
rate ratio were the same across different cauese of death. By that
token the facility is not likely to be implemented anytime soon, if
ever.

\section{Simulation based confidence intervals}

The parametric bootstrap is implemented in the function
\texttt{ci.Crisk} (\code{c}onfidence \code{i}ntervals for
\code{C}umulative \code{risk}s) in the \texttt{Epi} package:

We can now run the function using the model objects for the three
competing events, using a common prediction data frame, \texttt{nd}
for the rates. The time points in the frame must be so closely spaced
that it makes sense to assume the rates constant in each interval;
here we use intervals of length 1/20 years, in real applications we
would use 1/50 (about 1 week) or less:
\begin{Schunk}
\begin{Sinput}
> res <- ci.Crisk(list(OAD = mO,
+                      Ins = mI,
+                     Dead = mD),
+                             nd = data.frame(tfd = seq(0, 10, 1/20)),
+                             nB = 500,
+                           perm = 4:1)
\end{Sinput}
\begin{Soutput}
NOTE: Times are assumed to be in the column tfd at equal distances of 0.05 
\end{Soutput}
\begin{Sinput}
> str(res)
\end{Sinput}
\begin{Soutput}
List of 4
 $ Crisk: num [1:201, 1:4, 1:3] 1 0.959 0.923 0.892 0.864 ...
  ..- attr(*, "dimnames")=List of 3
  .. ..$ tfd  : chr [1:201] "0" "0.05" "0.1" "0.15" ...
  .. ..$ cause: chr [1:4] "Surv" "OAD" "Ins" "Dead"
  .. ..$      : chr [1:3] "50%" "2.5%" "97.5%"
 $ Srisk: num [1:201, 1:3, 1:3] 0 0.00342 0.00665 0.00972 0.01264 ...
  ..- attr(*, "dimnames")=List of 3
  .. ..$ tfd  : chr [1:201] "0" "0.05" "0.1" "0.15" ...
  .. ..$ cause: chr [1:3] "Dead" "Dead+Ins" "Dead+Ins+OAD"
  .. ..$      : chr [1:3] "50%" "2.5%" "97.5%"
 $ Stime: num [1:201, 1:4, 1:3] 0 0.049 0.096 0.141 0.185 ...
  ..- attr(*, "dimnames")=List of 3
  .. ..$ tfd  : chr [1:201] "0" "0.05" "0.1" "0.15" ...
  .. ..$ cause: chr [1:4] "Surv" "OAD" "Ins" "Dead"
  .. ..$      : chr [1:3] "50%" "2.5%" "97.5%"
 $ time : num [1:201] 0 0.05 0.1 0.15 0.2 0.25 0.3 0.35 0.4 0.45 ...
 - attr(*, "int")= num 0.05
\end{Soutput}
\end{Schunk}
As we see, the returned object (\texttt{res}) is a list of length 4,
the first 3 components are 3-way arrays, and the last the vector of
times of the first dimension of the arrays. The latter is mainly for
convenience in further processing---it is easier to write
\texttt{res\$time} than
\texttt{as.numeric(dimnames(res\$Crisk)[[1]])}.

The three first components of \texttt{res} represent:
\begin{itemize}
\item \texttt{Crisk}: \texttt{C}umulative \texttt{risk}s for each state
\item \texttt{Srisk}: \texttt{S}tacked cumulative \texttt{risk}s across
  states
\item \texttt{Stime}: \texttt{S}ojourn \texttt{time}s in each state,
  truncated at each point of the time dimension.
\end{itemize}

The first dimension of each array is time corresponding to endpoints
of intervals of length \texttt{int}, (normally assumed starting at 0,
but not necessarily). The second dimension is states (or combinations
thereof). The last dimension of the arrays is the type of statistic;
\texttt{50\%} is the median of the samples, and the bootstrap
confidence intervals as indicated; taken from the \texttt{alpha}
argument to \texttt{ci.Crisk} (defaults to 0.05).

The argument \texttt{perm} governs in which order the state
probabilities are stacked in the \texttt{Srisk} element of the
returned list, the default is the states in the order given in the
list of models in the first argument to \texttt{ci.Crisk} followed by
the survival.

If we want the bootstrap samples to make other calculations we can ask
the function to return the bootstrap samples of the rates by using the
argument \texttt{sim.res = 'rates'} (defaults to \texttt{'none'}):
\begin{Schunk}
\begin{Sinput}
> rsm <- ci.Crisk(list(OAD = mO,
+                      Ins = mI,
+                     Dead = mD),
+                             nd = data.frame(tfd = seq(0, 10, 1/20)),
+                             nB = 500,
+                        sim.res = 'rates')
\end{Sinput}
\begin{Soutput}
NOTE: Times are assumed to be in the column tfd at equal distances of 0.05 
\end{Soutput}
\begin{Sinput}
> str(rsm)
\end{Sinput}
\begin{Soutput}
 num [1:201, 1:3, 1:500] 0.464 0.442 0.422 0.402 0.383 ...
 - attr(*, "dimnames")=List of 3
  ..$ time: chr [1:201] "0" "0.05" "0.1" "0.15" ...
  ..$ mod : chr [1:3] "OAD" "Ins" "Dead"
  ..$ sim : chr [1:500] "1" "2" "3" "4" ...
 - attr(*, "int")= num 0.05
 - attr(*, "time")= num [1:201] 0 0.05 0.1 0.15 0.2 0.25 0.3 0.35 0.4 0.45 ...
\end{Soutput}
\end{Schunk}
This is 500 bootstrap samples (defined by \texttt{nB=}) of the rates
evaluated at the 201 endpoints of the intervals (defined in
\texttt{nd}).

Alternatively we can get the bootstrap samples of the cumulative risks
by setting \texttt{sim.res = 'crisk'}:
\begin{Schunk}
\begin{Sinput}
> csm <- ci.Crisk(list(OAD = mO,
+                      Ins = mI,
+                     Dead = mD),
+                             nd = data.frame(tfd = seq(0, 10, 1/20)),
+                             nB = 500,
+                        sim.res = 'crisk')
\end{Sinput}
\begin{Soutput}
NOTE: Times are assumed to be in the column tfd at equal distances of 0.05 
\end{Soutput}
\begin{Sinput}
> str(csm)
\end{Sinput}
\begin{Soutput}
 num [1:201, 1:4, 1:500] 1 0.959 0.922 0.89 0.862 ...
 - attr(*, "dimnames")=List of 3
  ..$ tfd  : chr [1:201] "0" "0.05" "0.1" "0.15" ...
  ..$ cause: chr [1:4] "Surv" "OAD" "Ins" "Dead"
  ..$ sim  : chr [1:500] "1" "2" "3" "4" ...
 - attr(*, "int")= num 0.05
\end{Soutput}
\end{Schunk}
These are 500 simulated samples of the cumulative risks evaluated at
the 201 endpoints of the intervals, and also includes the survival
probability in the first slot of the \nth{2} dimension of
\texttt{csm}.

\section{Simulated confidence intervals for rates}

In figure \ref{fig:rates} we showed the rates with confidence
intervals from the model. But in \texttt{rsm} we have 500
parametric bootstrap samples of the occurrence rates, so we can
derive the bootstrap medians and the bootstrap c.i.s:
\begin{Schunk}
\begin{Sinput}
> Brates <- aperm(apply(rsm,
+                       1:2,
+                       quantile,
+                       probs = c(.5, .025, .975)),
+                 c(2, 3, 1))
> str(Brates)
\end{Sinput}
\begin{Soutput}
 num [1:201, 1:3, 1:3] 0.462 0.439 0.418 0.398 0.379 ...
 - attr(*, "dimnames")=List of 3
  ..$ time: chr [1:201] "0" "0.05" "0.1" "0.15" ...
  ..$ mod : chr [1:3] "OAD" "Ins" "Dead"
  ..$     : chr [1:3] "50%" "2.5%" "97.5%"
\end{Soutput}
\end{Schunk}
(\texttt{aperm} permutes the dimensions of the array).
Then we can plot the bootstrap estimates on top of the estimates based
on the normal approximation to distribution of the parameters. They
are---not surprisingly---in close agreement since they are both based
on an assumption of normality of the parameters on the log-rate scale:

\insfig{rates-ci}{0.9}{Estimated rates from the \textrm{\tt DM} state,
  estimates are from \textrm{\tt gam} models fitted to data split in 1
  month intervals (1/12 year, that is). The white dotted curves are
  the bootstrap medians, black dotted curves are the bootstrap 95\%
  c.i.s.}
\begin{Schunk}
\begin{Sinput}
> matshade(nd$tfd, cbind(ci.pred(mD, nd),
+                        ci.pred(mI, nd),
+                        ci.pred(mO, nd)) * 1000,
+          ylim = c(0.1,500), yaxt = "n",
+          ylab = "Rates per 1000 PY",
+          xlab = "Time since DM diagnosis (years)",
+          col = c("black", "red", "blue"), log = "y", lwd = 3, plot = TRUE)
> matlines(nd$tfd,
+          cbind(Brates[, "Dead", ],
+                Brates[, "Ins" , ],
+                Brates[, "OAD" , ]) * 1000,
+          col = c("white", "black", "black"), lty = 3, lwd = c(3,1,1))
> axis(side = 2, at = ll <- outer(c(1,2,5), -2:3, function(x, y) x * 10^y),
+                labels = formatC(ll, digits = 4), las = 1)
> axis(side = 2, at = outer(c(1.5, 2:9), -2:3, function(x, y) x * 10^y),
+                labels = NA, tcl = -0.3)
> text(0, 0.5 * 0.6^c(1,2,0),
+      c("Dead", "Ins", "OAD"),
+      col = c("black", "red", "blue"), adj = 0, font = 2)
\end{Sinput}
\end{Schunk}

\section{Confidence intervals for cumulative risks}

In the \texttt{Crisk} component of \texttt{res} we have the
cumulative risks as functions of of time, with bootstrap confidence
intervals, so we can easily plot the three cumulative risks:

\insfig{crates}{0.9}{Cumulative risks for the three types of events,
  with 95\% bootstrap-based confidence intervals as shades.}
\begin{Schunk}
\begin{Sinput}
> matshade(res$time,
+          cbind(res$Crisk[,"Dead",],
+                res$Crisk[,"Ins" ,],
+                res$Crisk[,"OAD" ,]), plot = TRUE,
+          xlim = c(0,10), xaxs = "i", yaxs = "i", las = 1,
+          xlab = "Time since DM diagnosis (years)",
+          ylab = "Cumulative probability",
+           col = c("black","red","blue"))
> text(8, 0.3 + c(1, 0, 2) / 25,
+      c("Dead", "Ins", "OAD"),
+      col = c("black", "red", "blue"), adj = 0)
\end{Sinput}
\end{Schunk}

\section{Confidence intervals for stacked cumulative risks}

Unlike the single cumulative risks where we have a confidence
interval for each cumulative risk, when we want to show the stacked
probabilities we must deliver the confidence intervals for the
relevant sums, they are in the \texttt{Srisk} component of \texttt{res}.
\begin{Schunk}
\begin{Sinput}
> str(res$Crisk)
\end{Sinput}
\begin{Soutput}
 num [1:201, 1:4, 1:3] 1 0.959 0.923 0.892 0.864 ...
 - attr(*, "dimnames")=List of 3
  ..$ tfd  : chr [1:201] "0" "0.05" "0.1" "0.15" ...
  ..$ cause: chr [1:4] "Surv" "OAD" "Ins" "Dead"
  ..$      : chr [1:3] "50%" "2.5%" "97.5%"
\end{Soutput}
\begin{Sinput}
> str(res$Srisk)
\end{Sinput}
\begin{Soutput}
 num [1:201, 1:3, 1:3] 0 0.00342 0.00665 0.00972 0.01264 ...
 - attr(*, "dimnames")=List of 3
  ..$ tfd  : chr [1:201] "0" "0.05" "0.1" "0.15" ...
  ..$ cause: chr [1:3] "Dead" "Dead+Ins" "Dead+Ins+OAD"
  ..$      : chr [1:3] "50%" "2.5%" "97.5%"
\end{Soutput}
\end{Schunk}
But we start out by plotting the stacked probabilities using
\texttt{mat2pol} (\texttt{mat}rix to \texttt{pol}ygon), the input
required is the single components from the \texttt{Crisk}
component. Then we add the confidence intervals as white shades (using
\texttt{matshade}):

\insfig{stack-ci}{0.9}{Probabilities of being in the 4 different
  states as a function of time since diagnosis. Note that \textrm{\tt OAD}
  means that OAD was initiated first, and similarly for
  \textrm{\tt Ins}. We are not concerned about what occurs \emph{after}
  these events.  \textrm{\tt Dead} means dead without being on any
  drug.\newline The white shadings around the borders between coloured
  areas represent the 95\% confidence intervals for the (sum of)
  probabilities.}

\begin{Schunk}
\begin{Sinput}
> zz <- mat2pol(res$Crisk[,c("Dead", "Ins", "OAD", "Surv"),1],
+               x = res$time,
+            xlim = c(0, 10), xaxs = "i", yaxs = "i", las = 1,
+            xlab = "Time since DM diagnosis (years)",
+            ylab = "Probability",
+             col =  c("black","red","blue","forestgreen") )
> text(9, mp(zz["9",]), c("Dead", "Ins", "OAD", "DM"), col = "white" )
> matshade(res$time,
+          cbind(res$Srisk[, 1, ],
+                res$Srisk[, 2, ],
+                res$Srisk[, 3, ]),
+          col = 'transparent', col.shade = "white", alpha = 0.4)
\end{Sinput}
\end{Schunk}

\section{Sojourn times}

From the \texttt{Stime} component of the \texttt{res} we can derive
the estimated time spent in each state during the first, say, 5 or 10 years:

When referring to the times, we use \emph{character} values---5 and 10
years are not necessarily at the \nth{5} and \nth{10} positions of the
first dimension of the \texttt{Stime} array:
\begin{Schunk}
\begin{Sinput}
> s510 <- res$Stime[c("5", "10"),,]
> dimnames(s510)[[1]] <- c(" 5 yr","10 yr")
> round(ftable(s510, row.vars=1:2), 2)
\end{Sinput}
\begin{Soutput}
              50% 2.5% 97.5%
tfd   cause                 
 5 yr Surv   2.77 2.72  2.82
      OAD    1.44 1.40  1.50
      Ins    0.40 0.37  0.43
      Dead   0.39 0.36  0.42
10 yr Surv   4.32 4.22  4.42
      OAD    3.64 3.54  3.76
      Ins    0.84 0.78  0.90
      Dead   1.20 1.13  1.28
\end{Soutput}
\end{Schunk}
So we see that the expected life lived without pharmaceutical treatment
during the first 10 years after DM diagnosis is 4.31 years with a 95\%
CI of (4.21; 4.41), and during the first 5 years 2.77 (2.72; 2.82).

\chapter{A simple illustration of \texttt{ci.Crisk}}

The following is a terse cook-book illustration of how to use the
\texttt{ci.Crisk} function.

\section{Data}

For illustration we simulate some causes of death in the
\texttt{DMlate} data set; first we sample numbers 1, 2, 3 representing
3 different causes of death in \texttt{DMlate}:
\begin{Schunk}
\begin{Sinput}
> data(DMlate)
> set.seed(7465)
> wh <- sample(1:3, nrow(DMlate), replace = T, prob = c(4, 2, 6))
\end{Sinput}
\end{Schunk}
Those not dead are changed to 0:
\begin{Schunk}
\begin{Sinput}
> wh[is.na(DMlate$dodth)] <- 0
\end{Sinput}
\end{Schunk}
Define a factor in \texttt{DMlate} defining exit status as either alive or one
of the three causes of death, and check by a \texttt{table} that all
dead have a cause:
\begin{Schunk}
\begin{Sinput}
> DMlate$codth <- factor(wh, labels = c("Alive", "CVD", "Can", "Oth"))
> with(DMlate, table(codth, isDead = !is.na(dodth)))
\end{Sinput}
\begin{Soutput}
       isDead
codth   FALSE TRUE
  Alive  7497    0
  CVD       0  815
  Can       0  401
  Oth       0 1287
\end{Soutput}
\end{Schunk}
It is important that the \texttt{"Alive"} state is the \texttt{first}
level if the factor \texttt{codth}; the \texttt{Lexis} function will
assign this the all persons at start of follow-up.

\texttt{DMlate} now looks like a typical data set with cause of death
in a separate variable; in this case we also added a state, \texttt{Alive},
for those without a recorded death:
\begin{Schunk}
\begin{Sinput}
> str(DMlate)
\end{Sinput}
\begin{Soutput}
'data.frame':	10000 obs. of  8 variables:
 $ sex  : Factor w/ 2 levels "M","F": 2 1 2 2 1 2 1 1 2 1 ...
 $ dobth: num  1940 1939 1918 1965 1933 ...
 $ dodm : num  1999 2003 2005 2009 2009 ...
 $ dodth: num  NA NA NA NA NA ...
 $ dooad: num  NA 2007 NA NA NA ...
 $ doins: num  NA NA NA NA NA NA NA NA NA NA ...
 $ dox  : num  2010 2010 2010 2010 2010 ...
 $ codth: Factor w/ 4 levels "Alive","CVD",..: 1 1 1 1 1 4 1 1 4 1 ...
\end{Soutput}
\begin{Sinput}
> head(DMlate, 12)
\end{Sinput}
\begin{Soutput}
       sex    dobth     dodm    dodth    dooad doins      dox codth
50185    F 1940.256 1998.917       NA       NA    NA 2009.997 Alive
307563   M 1939.218 2003.309       NA 2007.446    NA 2009.997 Alive
294104   F 1918.301 2004.552       NA       NA    NA 2009.997 Alive
336439   F 1965.225 2009.261       NA       NA    NA 2009.997 Alive
245651   M 1932.877 2008.653       NA       NA    NA 2009.997 Alive
216824   F 1927.870 2007.886 2009.923       NA    NA 2009.923   Oth
24969    M 1946.498 2007.216       NA 2007.248    NA 2009.997 Alive
325465   M 1940.475 2006.674       NA 2006.674    NA 2009.997 Alive
430048   F 1937.083 1998.956 2008.464       NA    NA 2008.464   Oth
362980   M 1933.154 2009.784       NA       NA    NA 2009.997 Alive
279976   F 1928.300 2000.949       NA       NA    NA 2009.997 Alive
279271   M 1951.213 2008.114       NA 2009.743    NA 2009.997 Alive
\end{Soutput}
\end{Schunk}

\section{A \texttt{Lexis} object with 3 causes of death}

With cause of death in a separate variable it is easy to set up a
\texttt{Lexis} object:
\begin{Schunk}
\begin{Sinput}
> dmL <- Lexis(entry = list(per = dodm,
+                           age = dodm - dobth,
+                           tfD = 0),
+               exit = list(per = dox),
+        exit.status = codth,
+               data = DMlate)
\end{Sinput}
\begin{Soutput}
NOTE: entry.status has been set to "Alive" for all.
NOTE: Dropping  4  rows with duration of follow up < tol
\end{Soutput}
\begin{Sinput}
> summary(dmL, t = T)
\end{Sinput}
\begin{Soutput}
Transitions:
     To
From    Alive CVD Can  Oth  Records:  Events: Risk time:  Persons:
  Alive  7497 814 400 1285      9996     2499   54273.27      9996

Timescales:
per age tfD 
 ""  ""  "" 
\end{Soutput}
\end{Schunk}
We can show the overall rates (the default \texttt{boxes} is
\emph{very} primitive):
\begin{Schunk}
\begin{Sinput}
> boxes(dmL, boxpos = TRUE)
\end{Sinput}
\end{Schunk}
\insfig{boxes}{0.8}{Transitions from live to different causes of
  death. You probably want to explore the other arguments to
  \textrm{\tt boxes}.}

\section{Models for the rates}

In order to model the cause-specific mortality rates by sex and time from
diagnosis (\texttt{tfD}), we first split the data in 6-month intervals
\begin{Schunk}
\begin{Sinput}
> sL <- splitLexis(dmL, time.scale = "age", breaks = seq(0, 120, 1/2))
> summary(sL)
\end{Sinput}
\begin{Soutput}
Transitions:
     To
From     Alive CVD Can  Oth  Records:  Events: Risk time:  Persons:
  Alive 115974 814 400 1285    118473     2499   54273.27      9996
\end{Soutput}
\end{Schunk}
\begin{Schunk}
\begin{Sinput}
> mCVD <- gamLexis(sL, ~ s(tfD, by=sex), to = "CVD")
\end{Sinput}
\begin{Soutput}
mgcv::gam Poisson analysis of Lexis object sL with log link:
Rates for the transition:
Alive->CVD
\end{Soutput}
\begin{Sinput}
> mCan <- gamLexis(sL, ~ s(tfD, by=sex), to = "Can")
\end{Sinput}
\begin{Soutput}
mgcv::gam Poisson analysis of Lexis object sL with log link:
Rates for the transition:
Alive->Can
\end{Soutput}
\begin{Sinput}
> mOth <- gamLexis(sL, ~ s(tfD, by=sex), to = "Oth")
\end{Sinput}
\begin{Soutput}
mgcv::gam Poisson analysis of Lexis object sL with log link:
Rates for the transition:
Alive->Oth
\end{Soutput}
\end{Schunk}

\section{Derived measures}

With these three models for the occurrence rates we can compute the
cumulative risks of dying from each of the causes. We need a
prediction data frame that gives the rates at closely spaced times, in
this case for men. For women the code would be practically the same:
\begin{Schunk}
\begin{Sinput}
> nm <- data.frame(tfD = seq(0, 15, 1/20), sex = "M")
\end{Sinput}
\end{Schunk}
Note that we can rename the states as we please by naming the entries
in the list of models we supply to \texttt{ci.Crisk}:
\begin{Schunk}
\begin{Sinput}
> cR <- ci.Crisk(list(CVD = mCVD,
+                     Can = mCan,
+                   Other = mOth),
+                nB = 500,
+                nd = nm)
\end{Sinput}
\begin{Soutput}
NOTE: Times are assumed to be in the column tfD at equal distances of 0.05 
\end{Soutput}
\begin{Sinput}
> str(cR)
\end{Sinput}
\begin{Soutput}
List of 4
 $ Crisk: num [1:301, 1:4, 1:3] 1 0.997 0.993 0.99 0.987 ...
  ..- attr(*, "dimnames")=List of 3
  .. ..$ tfD  : chr [1:301] "0" "0.05" "0.1" "0.15" ...
  .. ..$ cause: chr [1:4] "Surv" "CVD" "Can" "Other"
  .. ..$      : chr [1:3] "50%" "2.5%" "97.5%"
 $ Srisk: num [1:301, 1:3, 1:3] 0 0.00179 0.00353 0.00521 0.00687 ...
  ..- attr(*, "dimnames")=List of 3
  .. ..$ tfD  : chr [1:301] "0" "0.05" "0.1" "0.15" ...
  .. ..$ cause: chr [1:3] "Other" "Other+Can" "Other+Can+CVD"
  .. ..$      : chr [1:3] "50%" "2.5%" "97.5%"
 $ Stime: num [1:301, 1:4, 1:3] 0 0.0499 0.0997 0.1492 0.1987 ...
  ..- attr(*, "dimnames")=List of 3
  .. ..$ tfD  : chr [1:301] "0" "0.05" "0.1" "0.15" ...
  .. ..$ cause: chr [1:4] "Surv" "CVD" "Can" "Other"
  .. ..$      : chr [1:3] "50%" "2.5%" "97.5%"
 $ time : num [1:301] 0 0.05 0.1 0.15 0.2 0.25 0.3 0.35 0.4 0.45 ...
 - attr(*, "int")= num 0.05
\end{Soutput}
\end{Schunk}
Note that we get three arrays: \texttt{Crisk}, the cumulative risks;
\texttt{Srisk}, the stacked risks and \texttt{Stime}, the sojourn
times in each state. Finally, for convenience we also have the
component \texttt{time}, the times at which the cumulative risks are
computed. It is also available as the clumpy expression
\texttt{as.numeric(dimnames(cR\$Crisk)[[1]])}, but \texttt{cR\$time}
is easier.

\subsection{Cumulative risks}

We can plot the cumulative risks for death from each of the three
causes, note we use the colors from last. Note that the time points
are in the \texttt{time} component of the \texttt{Crisk} object:
\begin{Schunk}
\begin{Sinput}
> clr <- c("black", "orange", "limegreen")
> matshade(cR$time, cbind(cR$Crisk[, "CVD"  , ],
+                         cR$Crisk[, "Can"  , ],
+                         cR$Crisk[, "Other", ]),
+          col = clr, lty = 1, lwd = 2,
+          plot = TRUE, ylim = c(0, 1/3), yaxs = "i")
> text(0, 1/3 - c(2,3,1)/30, c("CVD", "Can", "Oth"),
+      col = clr, adj = 0, font = 2)
\end{Sinput}
\end{Schunk}
\insfig{cR}{0.9}{Cumulative risks of each cause of death based on
\textrm{\tt gam} models for the cause-specific rates.}

We also have the stacked probabilities so we can show how the
population is distributed across the states at any one time:

\subsection{Stacked cumulative risks}

We also get the stacked probabilities in the order that we supplied the
models, so that if we plot these we get the probabilities of being
dead from each cause as the \emph{difference} between the curves. And
the confidence intervals are confidence intervals for the cumulative
sums of probabilities.
\begin{Schunk}
\begin{Sinput}
> matshade(cR$time, cbind(cR$Srisk[,1,],
+                         cR$Srisk[,2,],
+                         cR$Srisk[,3,]),
+          col = "black", lty = 1, lwd = 2,
+          plot = TRUE, ylim = c(0,1), xaxs = "i", yaxs = "i")
> text(14, mp(c(0, cR$Srisk["14", , 1], 1)),
+      rev(c(dimnames(cR$Crisk)[[2]])))
> box(bty = "o")
\end{Sinput}
\end{Schunk}
\insfig{Sr1}{0.9}{Stacked cumulative risks.}

It is not a good idea to color these curves, they do not refer to the
causes of death, it is the areas \emph{between} the curves that refer
to causes. By the same token, since the quantity of interest is the
area between the curves and horizontal lines at 0 and 1, it is
important that the horizontal axes are placed at precisely 0 and 1 on
the vertical axis. This is what \texttt{yaxs = "i"} achieves.

It would be more logical to color the areas \emph{between} the
curves. which can be done by \texttt{mat2pol} (\texttt{mat}rix to
\texttt{pol}ygons) using the \texttt{Crisk} component. We can then
superpose the confidence intervals for the sum of the state
probabilities using \texttt{matshade} by adding white shades:
\begin{Schunk}
\begin{Sinput}
> zz <- mat2pol(cR$Crisk[, c("Other", "Can", "CVD", "Surv"), "50%"],
+               x = cR$time,
+            xlim = c(0,15), xaxs = "i", yaxs = "i", las = 1,
+            xlab = "Time since DM diagnosis (years)",
+            ylab = "Probability",
+             col =  c("gray", "red", "blue", "limegreen") )
> matshade(cR$time, cbind(cR$Srisk[,1,],
+                         cR$Srisk[,2,],
+                         cR$Srisk[,3,]),
+          col = "transparent", col.shade = "white", alpha = 0.4)
> text(14, mp(c(0, cR$Srisk["14", , 1], 1)),
+      rev(c(dimnames(cR$Crisk)[[2]])), col = "white")
\end{Sinput}
\end{Schunk}
\insfig{Sr2}{0.9}{Stacked cumulative risks with coloring of states and
overlaid with confidence intervals for the probabilities shown; that
is the relevant sums.}

\subsection{Sojourn times}

The third component of the result, \texttt{Stime} is an array of
sojourn times over intervals starting at 0 and ending at the time
indicated by the first dimension:
\begin{Schunk}
\begin{Sinput}
> ftable(round(cR$Stime[paste(1:5 * 3), , ], 1), row.vars = 1)
\end{Sinput}
\begin{Soutput}
    cause Surv             CVD             Can            Other           
           50% 2.5% 97.5%  50% 2.5% 97.5%  50% 2.5% 97.5%   50% 2.5% 97.5%
tfD                                                                       
3          2.8  2.8   2.8  0.1  0.1   0.1  0.0  0.0   0.0   0.1  0.1   0.1
6          5.3  5.2   5.3  0.2  0.2   0.3  0.1  0.1   0.1   0.4  0.4   0.4
9          7.4  7.4   7.5  0.5  0.5   0.6  0.3  0.2   0.3   0.8  0.8   0.9
12         9.3  9.2   9.4  0.9  0.8   0.9  0.4  0.4   0.5   1.4  1.3   1.5
15        10.9 10.7  11.0  1.3  1.3   1.5  0.7  0.6   0.8   2.1  2.0   2.2
\end{Soutput}
\end{Schunk}
The sojourn times in the three dead states can be taken as the years
of life lost to each of the causes, the sum of the medians for the
three causes equals the time frame (5, 10, 15) minus the \texttt{Surv}
component.

So we see that during the first 15 years after diagnosis of diabetes,
the expected years alive is 10.9 years. The distribution of lifetime lost
between the causes is bogus in this case as the causes of death were
randomly generated.

\subsection{Comparing groups}

Finally, we may want to see the \emph{difference} (or ratio) of
survival probabilities between men and women, say. This can be derived
from two bootstrap samples using different prediction frames (the
argument \texttt{nd=} to \texttt{ci.Crisk}). But the two bootstrap
samples of parameters must be the same, i.e. come from the
\emph{same} stream of samples from the multivariate normal. This can
be obtained by explicitly setting the seed for the random number
generator to the same value before calling \texttt{ci.Crisk} with each
of the two different prediction frames as \texttt{nd} argument:
\begin{Schunk}
\begin{Sinput}
> nm <- data.frame(tfD = seq(0, 15, 1/20), sex = "M")
> nw <- data.frame(tfD = seq(0, 15, 1/20), sex = "F")
> # set the seed
> set.seed(1952)
> mR <- ci.Crisk(list(CVD = mCVD,
+                     Can = mCan,
+                   Other = mOth),
+                nd = nm,
+                nB = 500,
+           sim.res = "crisk" )
\end{Sinput}
\begin{Soutput}
NOTE: Times are assumed to be in the column tfD at equal distances of 0.05 
\end{Soutput}
\begin{Sinput}
> # REset the seed
> set.seed(1952)
> wR <- ci.Crisk(list(CVD = mCVD,
+                     Can = mCan,
+                   Other = mOth),
+                nd = nw,
+                nB = 500,
+           sim.res = "crisk" )
\end{Sinput}
\begin{Soutput}
NOTE: Times are assumed to be in the column tfD at equal distances of 0.05 
\end{Soutput}
\begin{Sinput}
> str(wR)
\end{Sinput}
\begin{Soutput}
 num [1:301, 1:4, 1:500] 1 0.997 0.994 0.991 0.988 ...
 - attr(*, "dimnames")=List of 3
  ..$ tfD  : chr [1:301] "0" "0.05" "0.1" "0.15" ...
  ..$ cause: chr [1:4] "Surv" "CVD" "Can" "Other"
  ..$ sim  : chr [1:500] "1" "2" "3" "4" ...
 - attr(*, "int")= num 0.05
\end{Soutput}
\end{Schunk}
The two samples are now from identical streams of random numbers, so
we can get differences and ratios of the survival curves between men
and women:
\begin{Schunk}
\begin{Sinput}
> dS <- mR[,"Surv",] - wR[,"Surv",]
> dS <- apply(dS, 1, quantile, probs = c(.5, .025, .975)) * 100
> str(dS)
\end{Sinput}
\begin{Soutput}
 num [1:3, 1:301] 0 0 0 -0.0342 -0.0865 ...
 - attr(*, "dimnames")=List of 2
  ..$    : chr [1:3] "50%" "2.5%" "97.5%"
  ..$ tfD: chr [1:301] "0" "0.05" "0.1" "0.15" ...
\end{Soutput}
\begin{Sinput}
> rS <- mR[,"Surv",] / wR[,"Surv",]
> rS <- apply(rS, 1, quantile, probs = c(.5, .025, .975))
\end{Sinput}
\end{Schunk}
We can then plot the differences and the ratios of the
probabilities---note that the dimension of the function
\texttt{apply}ed becomes the first dimension of the result:
\begin{Schunk}
\begin{Sinput}
> par(mfrow = c(1,2))
> matshade(as.numeric(colnames(dS)), t(dS), plot = TRUE,
+          lwd = 3, ylim = c(-5, 5),
+          xlab = "Time since DM diagnosis (years)",
+          ylab = "Men - Women survival difference (%)")
> abline(h = 0)
> matshade(as.numeric(colnames(rS)), t(rS), plot = TRUE,
+          lwd = 3, ylim = c(1/1.2, 1.2), log ="y",
+          xlab = "Time since DM diagnosis (years)",
+          ylab = "Men - Women survival ratio")
> abline(h = 1)
\end{Sinput}
\end{Schunk}
\insfig{difrat}{1.0}{Differences and ratios of survival between men
  and women, derived from the same set of bootstrap samples from the
  parameter vector.}

To illustrate the effect of \emph{not} pairing the random samples we
can generate a fresh sample for women from a different stream (by
\texttt{not} setting the seed) and do the calculations to illustrate
the excess we get from not aligning samples.
\begin{Schunk}
\begin{Sinput}
> fR <- ci.Crisk(list(CVD = mCVD,
+                     Can = mCan,
+                   Other = mOth),
+                nd = nw,
+                nB = 500,
+           sim.res = "crisk" )
\end{Sinput}
\begin{Soutput}
NOTE: Times are assumed to be in the column tfD at equal distances of 0.05 
\end{Soutput}
\begin{Sinput}
> dxS <- mR[,"Surv",] - fR[,"Surv",]
> dxS <- apply(dxS, 1, quantile, probs = c(.5, .025, .975)) * 100
> rxS <- mR[,"Surv",] / fR[,"Surv",]
> rxS <- apply(rxS, 1, quantile, probs = c(.5, .025, .975))
\end{Sinput}
\end{Schunk}
\begin{Schunk}
\begin{Sinput}
> par(mfrow = c(1,2))
> matshade(as.numeric(colnames(dS)), t(dS), plot = TRUE,
+          lwd = 3, ylim = c(-5, 5),
+          xlab = "Time since DM diagnosis (years)",
+          ylab = "Men - Women survival difference (%)")
> matshade(as.numeric(colnames(dxS)), t(dxS), lty = 3, col = "forestgreen")
> abline(h = 0)
> matshade(as.numeric(colnames(rS)), t(rS), plot = TRUE,
+          lwd = 3, ylim = c(1/1.2, 1.2), log ="y",
+          xlab = "Time since DM diagnosis (years)",
+          ylab = "Men - Women survival ratio")
> matshade(as.numeric(colnames(rxS)), t(rxS), lty = 3, col = "forestgreen")
> abline(h = 1)
\end{Sinput}
\end{Schunk}
\insfig{difratx}{1.0}{Differences and ratios of survival between men
  and women, derived from separate bootstrap samples. The outer
  confidence bands are from bootstrap samples not properly paired
  between men end women.}

\begin{Schunk}
\begin{Soutput}
  Start time: 2025-07-08, 12:26:55 
    End time: 2025-07-08, 12:28:22 
Elapsed time: 1.45 minutes
\end{Soutput}
\end{Schunk}

\end{document}
