
%\VignetteIndexEntry{Follow-up data with the Epi package}
\documentclass[a4paper,twoside,12pt]{article}

\usepackage[english]{babel}
\usepackage{booktabs,rotating,graphicx,amsmath,verbatim,fancyhdr,Sweave}
\usepackage[colorlinks,linkcolor=red,urlcolor=blue]{hyperref}
\newcommand{\R}{\textsf{\bf R}}
\renewcommand{\topfraction}{0.95}
\renewcommand{\bottomfraction}{0.95}
\renewcommand{\textfraction}{0.1}
\renewcommand{\floatpagefraction}{0.9}
\DeclareGraphicsExtensions{.pdf,.jpg}
\setcounter{secnumdepth}{1}
\setcounter{tocdepth}{1}

\oddsidemargin 1mm
\evensidemargin 1mm
\textwidth 160mm
\textheight 230mm
\topmargin -5mm
\headheight 8mm
\headsep 5mm
\footskip 15mm

\begin{document}

\raggedleft
\pagestyle{empty}
\vspace*{0.1\textheight}
\Huge
{\bf Follow-up data with the\\ \texttt{Epi} package}
\noindent\rule[-1ex]{\textwidth}{5pt}\\[2.5ex]
\Large
Spring 2009.
\vfill
\normalsize
\begin{tabular}{rl}
 Michael Hills & Retired \\
               & Highgate, London \\[1em]
Martyn Plummer & International Agency for Research on Cancer, Lyon\\
               & \texttt{plummer@iarc.fr} \\[1em]
Bendix Carstensen & Steno Diabetes Center, Gentofte, Denmark\\
                  & \small \& Department of Biostatistics,
                    University of Copenhagen\\
                  & \normalsize \texttt{bxc@steno.dk} \\
                  & \url{www.pubhealth.ku.dk/~bxc}
\end{tabular}
\normalsize
\newpage
\raggedright
\parindent 3ex
\parskip 0ex
\tableofcontents
\cleardoublepage
\setcounter{page}{1}
\pagestyle{fancy}
\renewcommand{\sectionmark}[1]{\markboth{\thesection #1}{\thesection \ #1}}
\fancyhead[OL]{\sl Follow-up data with the \texttt{Epi} package.}
\fancyhead[ER]{\sl \rightmark}
\fancyhead[EL,OR]{\bf \thepage}
\fancyfoot{}
\renewcommand{\headrulewidth}{0.1pt}

\begin{Schunk}
\begin{Sinput}
> library(Epi)
> print( sessionInfo(), l=F )
\end{Sinput}
\begin{Soutput}
R version 3.1.0 (2014-04-10)
Platform: i386-w64-mingw32/i386 (32-bit)

attached base packages:
[1] utils     datasets  graphics  grDevices stats     methods   base     

other attached packages:
[1] Epi_1.1.64     foreign_0.8-61

loaded via a namespace (and not attached):
[1] tools_3.1.0
\end{Soutput}
\end{Schunk}

\section{Follow-up data in the \texttt{Epi} package}

In the \texttt{Epi}-package, follow-up data is represented by adding
some extra variables to a dataframe. Such a dataframe is called a
\texttt{Lexis} object. The tools for handling follow-up data then use
the structure of this for special plots, tabulations etc.

Follow-up data basically consists of a time of entry, a time of exit
and an indication of the status at exit (normally either ``alive'' or
``dead''). Implicitly is also assumed a status \emph{during} the
follow-up (usually ``alive'').

\begin{figure}[htbp]
  \centering
\setlength{\unitlength}{1pt}
\begin{picture}(210,70)(0,75)
%\scriptsize
\thicklines
 \put(  0,80){\makebox(0,0)[r]{Age-scale}}
 \put( 50,80){\line(1,0){150}}
 \put( 50,80){\line(0,1){5}}
 \put(100,80){\line(0,1){5}}
 \put(150,80){\line(0,1){5}}
 \put(200,80){\line(0,1){5}}
 \put( 50,77){\makebox(0,0)[t]{35}}
 \put(100,77){\makebox(0,0)[t]{40}}
 \put(150,77){\makebox(0,0)[t]{45}}
 \put(200,77){\makebox(0,0)[t]{50}}

 \put(  0,115){\makebox(0,0)[r]{Follow-up}}

 \put( 80,105){\makebox(0,0)[r]{\small Two}}
 \put( 90,105){\line(1,0){87}}
 \put( 90,100){\line(0,1){10}}
 \put(100,100){\line(0,1){10}}
 \put(150,100){\line(0,1){10}}
 \put(180,105){\circle{6}}
 \put( 95,110){\makebox(0,0)[b]{1}}
 \put(125,110){\makebox(0,0)[b]{5}}
 \put(165,110){\makebox(0,0)[b]{3}}

 \put( 50,130){\makebox(0,0)[r]{\small One}}
 \put( 60,130){\line(1,0){70}}
 \put( 60,125){\line(0,1){10}}
 \put(100,125){\line(0,1){10}}
 \put(130,130){\circle*{6}}
 \put( 80,135){\makebox(0,0)[b]{4}}
 \put(115,135){\makebox(0,0)[b]{3}}
\end{picture}
  \caption{\it Follow-up of two persons}
  \label{fig:fu2}
\end{figure}

\section{Timescales}

A timescale is a variable that varies deterministically \emph{within}
each person during follow-up, \textit{e.g.}:
\begin{itemize}
  \item Age
  \item Calendar time
  \item Time since treatment
  \item Time since relapse
\end{itemize}
All timescales advance at the same pace, so the time followed is the
same on all timescales. Therefore, it suffices to use only the entry
point on each of the time scale, for example:
\begin{itemize}
  \item Age at entry.
  \item Date of entry.
  \item Time since treatment (\emph{at} treatment this is 0).
  \item Time since relapse (\emph{at} relapse this is 0)..
\end{itemize}
In the \texttt{Epi} package, follow-up in a cohort is represented in a
\texttt{Lexis} object.  A \texttt{Lexis} object is a dataframe with a
bit of extra structure representing the follow-up. For the
\texttt{nickel} data we would construct a \texttt{Lexis} object by:
\begin{Schunk}
\begin{Sinput}
> data( nickel )
> nicL <- Lexis( entry = list( per=agein+dob,
+                              age=agein,
+                              tfh=agein-age1st ),
+                 exit = list( age=ageout ),
+          exit.status = ( icd %in% c(162,163) )*1,
+                 data = nickel )
\end{Sinput}
\end{Schunk}
The \texttt{entry} argument is a \emph{named} list with the entry
points on each of the timescales we want to use. It defines the names
of the timescales and the entry points. The \texttt{exit} argument
gives the exit time on \emph{one} of the timescales, so the name of
the element in this list must match one of the neames of the
\texttt{entry} list. This is sufficient, because the follow-up time on
all time scales is the same, in this case \texttt{ageout - agein}. Now
take a look at the result:
\begin{Schunk}
\begin{Sinput}
> str( nickel )
\end{Sinput}
\begin{Soutput}
'data.frame':	679 obs. of  7 variables:
 $ id      : num  3 4 6 8 9 10 15 16 17 18 ...
 $ icd     : num  0 162 163 527 150 163 334 160 420 12 ...
 $ exposure: num  5 5 10 9 0 2 0 0.5 0 0 ...
 $ dob     : num  1889 1886 1881 1886 1880 ...
 $ age1st  : num  17.5 23.2 25.2 24.7 30 ...
 $ agein   : num  45.2 48.3 53 47.9 54.7 ...
 $ ageout  : num  93 63.3 54.2 69.7 76.8 ...
\end{Soutput}
\begin{Sinput}
> str( nicL )
\end{Sinput}
\begin{Soutput}
Classes 'Lexis' and 'data.frame':	679 obs. of  14 variables:
 $ per     : num  1934 1934 1934 1934 1934 ...
 $ age     : num  45.2 48.3 53 47.9 54.7 ...
 $ tfh     : num  27.7 25.1 27.7 23.2 24.8 ...
 $ lex.dur : num  47.75 15 1.17 21.77 22.1 ...
 $ lex.Cst : num  0 0 0 0 0 0 0 0 0 0 ...
 $ lex.Xst : num  0 1 1 0 0 1 0 0 0 0 ...
 $ lex.id  : int  1 2 3 4 5 6 7 8 9 10 ...
 $ id      : num  3 4 6 8 9 10 15 16 17 18 ...
 $ icd     : num  0 162 163 527 150 163 334 160 420 12 ...
 $ exposure: num  5 5 10 9 0 2 0 0.5 0 0 ...
 $ dob     : num  1889 1886 1881 1886 1880 ...
 $ age1st  : num  17.5 23.2 25.2 24.7 30 ...
 $ agein   : num  45.2 48.3 53 47.9 54.7 ...
 $ ageout  : num  93 63.3 54.2 69.7 76.8 ...
 - attr(*, "time.scales")= chr  "per" "age" "tfh"
 - attr(*, "time.since")= chr  "" "" ""
 - attr(*, "breaks")=List of 3
  ..$ per: NULL
  ..$ age: NULL
  ..$ tfh: NULL
\end{Soutput}
\begin{Sinput}
> head( nicL )
\end{Sinput}
\begin{Soutput}
       per     age     tfh lex.dur lex.Cst lex.Xst lex.id id icd exposure
1 1934.246 45.2273 27.7465 47.7535       0       0      1  3   0        5
2 1934.246 48.2684 25.0820 15.0028       0       1      2  4 162        5
3 1934.246 52.9917 27.7465  1.1727       0       1      3  6 163       10
4 1934.246 47.9067 23.1861 21.7727       0       0      4  8 527        9
5 1934.246 54.7465 24.7890 22.0977       0       0      5  9 150        0
6 1934.246 44.3314 23.0437 18.2099       0       1      6 10 163        2
       dob  age1st   agein  ageout
1 1889.019 17.4808 45.2273 92.9808
2 1885.978 23.1864 48.2684 63.2712
3 1881.255 25.2452 52.9917 54.1644
4 1886.340 24.7206 47.9067 69.6794
5 1879.500 29.9575 54.7465 76.8442
6 1889.915 21.2877 44.3314 62.5413
\end{Soutput}
\end{Schunk}
The \texttt{Lexis} object \texttt{nicL} has a variable for each
timescale which is the entry point on this timescale. The follow-up
time is in the variable \texttt{lex.dur} (\textbf{dur}ation).

There is a \texttt{summary} function for \texttt{Lexis} objects that
list the numer of transitions and records as well as the total
follow-up time:
\begin{Schunk}
\begin{Sinput}
> summary( nicL )
\end{Sinput}
\begin{Soutput}
Transitions:
     To
From   0   1  Records:  Events: Risk time:  Persons:
   0 542 137       679      137   15348.06       679
\end{Soutput}
\end{Schunk}
We defined the exit status to be death from lung cancer (ICD7
162,163), i.e. this variable is 1 if follow-up ended with a death from
this cause. If follow-up ended alive or by death from another cause,
the exit status is coded 0, i.e. as a censoring.

Note that the exit status is in the variable \texttt{lex.Xst}
(e\textbf{X}it \textbf{st}atus. The variable \texttt{lex.Cst} is the
state where the follow-up takes place (\textbf{C}urrent
\textbf{st}atus), in this case 0 (alive).

It is possible to get a visualization of the follow-up along the
timescales chosen by using the \texttt{plot} method for \texttt{Lexis}
objects. \texttt{nicL} is an object of \emph{class} \texttt{Lexis}, so
using the function \texttt{plot()} on it means that \R\ will look for
the function \texttt{plot.Lexis} and use this function.
\begin{Schunk}
\begin{Sinput}
> plot( nicL )
\end{Sinput}
\end{Schunk}
The function allows a lot of control over the output, and a
\texttt{points.Lexis} function allows plotting of the endpoints of
follow-up:
\begin{Schunk}
\begin{Sinput}
> par( mar=c(3,3,1,1), mgp=c(3,1,0)/1.6 )
> plot( nicL, 1:2, lwd=1, col=c("blue","red")[(nicL$exp>0)+1],
+       grid=TRUE, lty.grid=1, col.grid=gray(0.7),
+       xlim=1900+c(0,90), xaxs="i",
+       ylim=  10+c(0,90), yaxs="i", las=1 )
> points( nicL, 1:2, pch=c(NA,3)[nicL$lex.Xst+1],
+         col="lightgray", lwd=3, cex=1.5 )
> points( nicL, 1:2, pch=c(NA,3)[nicL$lex.Xst+1],
+         col=c("blue","red")[(nicL$exp>0)+1], lwd=1, cex=1.5 )
\end{Sinput}
\end{Schunk}
The results of these two plotting commands are in figure \ref{fig:Lexis-diagram}.
\begin{figure}[tb]
\centering
\label{fig:Lexis-diagram}
\includegraphics[width=0.39\textwidth]{Follow-up-nicL1}
\includegraphics[width=0.59\textwidth]{Follow-up-nicL2}
\caption{\it Lexis diagram of the \texttt{nickel} dataset, left panel
  the default version, the right one with bells
  and whistles. The red lines are for persons with exposure$>0$, so it
  is pretty evident that the oldest ones are the exposed part of the
  cohort.}
\end{figure}

\section{Splitting the follow-up time along a timescale}

The follow-up time in a cohort can be subdivided by for example
current age. This is achieved by the \texttt{splitLexis} (note that it
is \emph{not} called \texttt{split.Lexis}). This requires that the
timescale and the breakpoints on this timescale are supplied. Try:
\begin{Schunk}
\begin{Sinput}
> nicS1 <- splitLexis( nicL, "age", breaks=seq(0,100,10) )
> summary( nicL )
\end{Sinput}
\begin{Soutput}
Transitions:
     To
From   0   1  Records:  Events: Risk time:  Persons:
   0 542 137       679      137   15348.06       679
\end{Soutput}
\begin{Sinput}
> summary( nicS1 )
\end{Sinput}
\begin{Soutput}
Transitions:
     To
From    0   1  Records:  Events: Risk time:  Persons:
   0 2073 137      2210      137   15348.06       679
\end{Soutput}
\end{Schunk}
So we see that the number of events and the amount of follow-up is the
same in the two datasets; only the number of records differ.

To see how records are split for each individual, it is useful to list
the results for a few individuals:
\begin{Schunk}
\begin{Sinput}
> round( subset( nicS1, id %in% 8:10 ), 2 )
\end{Sinput}
\begin{Soutput}
   lex.id     per   age   tfh lex.dur lex.Cst lex.Xst id icd exposure     dob
11      4 1934.25 47.91 23.19    2.09       0       0  8 527        9 1886.34
12      4 1936.34 50.00 25.28   10.00       0       0  8 527        9 1886.34
13      4 1946.34 60.00 35.28    9.68       0       0  8 527        9 1886.34
14      5 1934.25 54.75 24.79    5.25       0       0  9 150        0 1879.50
15      5 1939.50 60.00 30.04   10.00       0       0  9 150        0 1879.50
16      5 1949.50 70.00 40.04    6.84       0       0  9 150        0 1879.50
17      6 1934.25 44.33 23.04    5.67       0       0 10 163        2 1889.91
18      6 1939.91 50.00 28.71   10.00       0       0 10 163        2 1889.91
19      6 1949.91 60.00 38.71    2.54       0       1 10 163        2 1889.91
   age1st agein ageout
11  24.72 47.91  69.68
12  24.72 47.91  69.68
13  24.72 47.91  69.68
14  29.96 54.75  76.84
15  29.96 54.75  76.84
16  29.96 54.75  76.84
17  21.29 44.33  62.54
18  21.29 44.33  62.54
19  21.29 44.33  62.54
\end{Soutput}
\end{Schunk}
The resulting object, \texttt{nicS1}, is again a \texttt{Lexis}
object, and so follow-up may be split further along another
timescale. Try this and list the results for individuals 8, 9 and 10 again:
\begin{Schunk}
\begin{Sinput}
> nicS2 <- splitLexis( nicS1, "tfh", breaks=c(0,1,5,10,20,30,100) )
> round( subset( nicS2, id %in% 8:10 ), 2 )
\end{Sinput}
\begin{Soutput}
   lex.id     per   age   tfh lex.dur lex.Cst lex.Xst id icd exposure     dob
13      4 1934.25 47.91 23.19    2.09       0       0  8 527        9 1886.34
14      4 1936.34 50.00 25.28    4.72       0       0  8 527        9 1886.34
15      4 1941.06 54.72 30.00    5.28       0       0  8 527        9 1886.34
16      4 1946.34 60.00 35.28    9.68       0       0  8 527        9 1886.34
17      5 1934.25 54.75 24.79    5.21       0       0  9 150        0 1879.50
18      5 1939.46 59.96 30.00    0.04       0       0  9 150        0 1879.50
19      5 1939.50 60.00 30.04   10.00       0       0  9 150        0 1879.50
20      5 1949.50 70.00 40.04    6.84       0       0  9 150        0 1879.50
21      6 1934.25 44.33 23.04    5.67       0       0 10 163        2 1889.91
22      6 1939.91 50.00 28.71    1.29       0       0 10 163        2 1889.91
23      6 1941.20 51.29 30.00    8.71       0       0 10 163        2 1889.91
24      6 1949.91 60.00 38.71    2.54       0       1 10 163        2 1889.91
   age1st agein ageout
13  24.72 47.91  69.68
14  24.72 47.91  69.68
15  24.72 47.91  69.68
16  24.72 47.91  69.68
17  29.96 54.75  76.84
18  29.96 54.75  76.84
19  29.96 54.75  76.84
20  29.96 54.75  76.84
21  21.29 44.33  62.54
22  21.29 44.33  62.54
23  21.29 44.33  62.54
24  21.29 44.33  62.54
\end{Soutput}
\end{Schunk}
If we want to model the effect of these timescales we will for each
interval use either the value of the left endpoint in each interval or
the middle. There is a function \texttt{timeBand} which returns these.
Try:
\begin{Schunk}
\begin{Sinput}
> timeBand( nicS2, "age", "middle" )[1:20]
\end{Sinput}
\begin{Soutput}
 [1] 45 45 55 65 75 85 95 45 55 55 65 55 45 55 55 65 55 55 65 75
\end{Soutput}
\begin{Sinput}
> # For nice printing and column labelling use the data.frame() function:
> data.frame( nicS2[,c("id","lex.id","per","age","tfh","lex.dur")],
+             mid.age=timeBand( nicS2, "age", "middle" ),
+             mid.tfh=timeBand( nicS2, "tfh", "middle" ) )[1:20,]
\end{Sinput}
\begin{Soutput}
   id lex.id      per     age     tfh lex.dur mid.age mid.tfh
1   3      1 1934.246 45.2273 27.7465  2.2535      45      25
2   3      1 1936.500 47.4808 30.0000  2.5192      45      65
3   3      1 1939.019 50.0000 32.5192 10.0000      55      65
4   3      1 1949.019 60.0000 42.5192 10.0000      65      65
5   3      1 1959.019 70.0000 52.5192 10.0000      75      65
6   3      1 1969.019 80.0000 62.5192 10.0000      85      65
7   3      1 1979.019 90.0000 72.5192  2.9808      95      65
8   4      2 1934.246 48.2684 25.0820  1.7316      45      25
9   4      2 1935.978 50.0000 26.8136  3.1864      55      25
10  4      2 1939.164 53.1864 30.0000  6.8136      55      65
11  4      2 1945.978 60.0000 36.8136  3.2712      65      65
12  6      3 1934.246 52.9917 27.7465  1.1727      55      25
13  8      4 1934.246 47.9067 23.1861  2.0933      45      25
14  8      4 1936.340 50.0000 25.2794  4.7206      55      25
15  8      4 1941.060 54.7206 30.0000  5.2794      55      65
16  8      4 1946.340 60.0000 35.2794  9.6794      65      65
17  9      5 1934.246 54.7465 24.7890  5.2110      55      25
18  9      5 1939.457 59.9575 30.0000  0.0425      55      65
19  9      5 1939.500 60.0000 30.0425 10.0000      65      65
20  9      5 1949.500 70.0000 40.0425  6.8442      75      65
\end{Soutput}
\end{Schunk}
Note that these are the midpoints of the intervals defined by
\texttt{breaks=}, \emph{not} the midpoints of the actual follow-up
intervals. This is because the variable to be used in modelling must
be independent of the consoring and mortality pattern --- it should
only depend on the chosen grouping of the timescale.

\section{Splitting time at a specific date}

If we have a recording of the date of a specific event as for example
recovery or relapse, we may classify follow-up time as being before of
after this intermediate event. This is achieved with the function
\texttt{cutLexis}, which takes three arguments: the time point, the
timescale, and the value of the (new) state following the date.

Now we define the age for the nickel vorkers where the cumulative
exposure exceeds 50 exposure years:
\begin{Schunk}
\begin{Sinput}
> subset( nicL, id %in% 8:10 )
\end{Sinput}
\begin{Soutput}
       per     age     tfh lex.dur lex.Cst lex.Xst lex.id id icd exposure
4 1934.246 47.9067 23.1861 21.7727       0       0      4  8 527        9
5 1934.246 54.7465 24.7890 22.0977       0       0      5  9 150        0
6 1934.246 44.3314 23.0437 18.2099       0       1      6 10 163        2
       dob  age1st   agein  ageout
4 1886.340 24.7206 47.9067 69.6794
5 1879.500 29.9575 54.7465 76.8442
6 1889.915 21.2877 44.3314 62.5413
\end{Soutput}
\begin{Sinput}
> agehi <- nicL$age1st + 50 / nicL$exposure
> nicC <- cutLexis( data=nicL, cut=agehi, timescale="age",
+                   new.state=2, precursor.states=0 )
> subset( nicC, id %in% 8:10 )
\end{Sinput}
\begin{Soutput}
          per     age     tfh lex.dur lex.Cst lex.Xst lex.id id icd exposure
4100 1934.246 47.9067 23.1861 21.7727       2       2      4  8 527        9
5    1934.246 54.7465 24.7890 22.0977       0       0      5  9 150        0
6    1934.246 44.3314 23.0437  1.9563       0       2      6 10 163        2
680  1936.203 46.2877 25.0000 16.2536       2       1      6 10 163        2
          dob  age1st   agein  ageout
4100 1886.340 24.7206 47.9067 69.6794
5    1879.500 29.9575 54.7465 76.8442
6    1889.915 21.2877 44.3314 62.5413
680  1889.915 21.2877 44.3314 62.5413
\end{Soutput}
\end{Schunk}
(The \texttt{precursor.states=} argument is explained below).
Note that individual 6 has had his follow-up split at age 25 where 50
exposure-years were attained. This could also have been achieved in
the split dataset \texttt{nicS2} instead of \texttt{nicL}, try:
\begin{Schunk}
\begin{Sinput}
> subset( nicS2, id %in% 8:10 )
\end{Sinput}
\begin{Soutput}
   lex.id      per     age     tfh lex.dur lex.Cst lex.Xst id icd exposure
13      4 1934.246 47.9067 23.1861  2.0933       0       0  8 527        9
14      4 1936.340 50.0000 25.2794  4.7206       0       0  8 527        9
15      4 1941.060 54.7206 30.0000  5.2794       0       0  8 527        9
16      4 1946.340 60.0000 35.2794  9.6794       0       0  8 527        9
17      5 1934.246 54.7465 24.7890  5.2110       0       0  9 150        0
18      5 1939.457 59.9575 30.0000  0.0425       0       0  9 150        0
19      5 1939.500 60.0000 30.0425 10.0000       0       0  9 150        0
20      5 1949.500 70.0000 40.0425  6.8442       0       0  9 150        0
21      6 1934.246 44.3314 23.0437  5.6686       0       0 10 163        2
22      6 1939.915 50.0000 28.7123  1.2877       0       0 10 163        2
23      6 1941.203 51.2877 30.0000  8.7123       0       0 10 163        2
24      6 1949.915 60.0000 38.7123  2.5413       0       1 10 163        2
        dob  age1st   agein  ageout
13 1886.340 24.7206 47.9067 69.6794
14 1886.340 24.7206 47.9067 69.6794
15 1886.340 24.7206 47.9067 69.6794
16 1886.340 24.7206 47.9067 69.6794
17 1879.500 29.9575 54.7465 76.8442
18 1879.500 29.9575 54.7465 76.8442
19 1879.500 29.9575 54.7465 76.8442
20 1879.500 29.9575 54.7465 76.8442
21 1889.915 21.2877 44.3314 62.5413
22 1889.915 21.2877 44.3314 62.5413
23 1889.915 21.2877 44.3314 62.5413
24 1889.915 21.2877 44.3314 62.5413
\end{Soutput}
\begin{Sinput}
> agehi <- nicS2$age1st + 50 / nicS2$exposure
> nicS2C <- cutLexis( data=nicS2, cut=agehi, timescale="age",
+                     new.state=2, precursor.states=0 )
> subset( nicS2C, id %in% 8:10 )
\end{Sinput}
\begin{Soutput}
     lex.id      per     age     tfh lex.dur lex.Cst lex.Xst id icd exposure
3142      4 1934.246 47.9067 23.1861  2.0933       2       2  8 527        9
3143      4 1936.340 50.0000 25.2794  4.7206       2       2  8 527        9
3144      4 1941.060 54.7206 30.0000  5.2794       2       2  8 527        9
3145      4 1946.340 60.0000 35.2794  9.6794       2       2  8 527        9
17        5 1934.246 54.7465 24.7890  5.2110       0       0  9 150        0
18        5 1939.457 59.9575 30.0000  0.0425       0       0  9 150        0
19        5 1939.500 60.0000 30.0425 10.0000       0       0  9 150        0
20        5 1949.500 70.0000 40.0425  6.8442       0       0  9 150        0
21        6 1934.246 44.3314 23.0437  1.9563       0       2 10 163        2
3150      6 1936.203 46.2877 25.0000  3.7123       2       2 10 163        2
3151      6 1939.915 50.0000 28.7123  1.2877       2       2 10 163        2
3152      6 1941.203 51.2877 30.0000  8.7123       2       2 10 163        2
3153      6 1949.915 60.0000 38.7123  2.5413       2       1 10 163        2
          dob  age1st   agein  ageout
3142 1886.340 24.7206 47.9067 69.6794
3143 1886.340 24.7206 47.9067 69.6794
3144 1886.340 24.7206 47.9067 69.6794
3145 1886.340 24.7206 47.9067 69.6794
17   1879.500 29.9575 54.7465 76.8442
18   1879.500 29.9575 54.7465 76.8442
19   1879.500 29.9575 54.7465 76.8442
20   1879.500 29.9575 54.7465 76.8442
21   1889.915 21.2877 44.3314 62.5413
3150 1889.915 21.2877 44.3314 62.5413
3151 1889.915 21.2877 44.3314 62.5413
3152 1889.915 21.2877 44.3314 62.5413
3153 1889.915 21.2877 44.3314 62.5413
\end{Soutput}
\end{Schunk}
Note that follow-up subsequent to the event is classified as being
in state 2, but that the final transition to state 1 (death from lung
cancer) is preserved. This is the point of the \texttt{precursor.states=}
argument. It names the states (in this case 0, ``Alive'') that will be
over-witten by \texttt{new.state} (in this case state 2, ``High
exposure''). Clearly, state 1 (``Dead'') should not be updated even if
it is after the time where the persons moves to state 2. In other
words, only state 0 is a precursor to state 2, state 1 is always
subsequent to state 2.

Note if the intermediate event is to be used as a time-dependent
variable in a Cox-model, then \texttt{lex.Cst} should be used as the
time-dependent variable, and \texttt{lex.Xst==1} as the event.

\section{Competing risks --- multiple types of events}

If we want to consider death from lung cancer and death from other
causes as separate events we can code these as for example 1 and 2.
\begin{Schunk}
\begin{Sinput}
> data( nickel )
> nicL <- Lexis( entry = list( per=agein+dob,
+                              age=agein,
+                              tfh=agein-age1st ),
+                 exit = list( age=ageout ),
+          exit.status = ( icd > 0 ) + ( icd %in% c(162,163) ),
+                 data = nickel )
> summary( nicL )
\end{Sinput}
\begin{Soutput}
Transitions:
     To
From  0   1   2  Records:  Events: Risk time:  Persons:
   0 47 495 137       679      632   15348.06       679
\end{Soutput}
\begin{Sinput}
> subset( nicL, id %in% 8:10 )
\end{Sinput}
\begin{Soutput}
       per     age     tfh lex.dur lex.Cst lex.Xst lex.id id icd exposure
4 1934.246 47.9067 23.1861 21.7727       0       1      4  8 527        9
5 1934.246 54.7465 24.7890 22.0977       0       1      5  9 150        0
6 1934.246 44.3314 23.0437 18.2099       0       2      6 10 163        2
       dob  age1st   agein  ageout
4 1886.340 24.7206 47.9067 69.6794
5 1879.500 29.9575 54.7465 76.8442
6 1889.915 21.2877 44.3314 62.5413
\end{Soutput}
\end{Schunk}
If we want to label the states, we can enter the names of these in the
\texttt{states} parameter, try for example:
\begin{Schunk}
\begin{Sinput}
> nicL <- Lexis( entry = list( per=agein+dob,
+                              age=agein,
+                              tfh=agein-age1st ),
+                 exit = list( age=ageout ),
+          exit.status = ( icd > 0 ) + ( icd %in% c(162,163) ),
+                 data = nickel,
+               states = c("Alive","D.oth","D.lung") )
> summary( nicL )
\end{Sinput}
\begin{Soutput}
Transitions:
     To
From    Alive D.oth D.lung  Records:  Events: Risk time:  Persons:
  Alive    47   495    137       679      632   15348.06       679
\end{Soutput}
\end{Schunk}

Note that the \texttt{Lexis} function automatically assumes that all
persons enter in the first level (given in the \texttt{states=}
argument)

When we cut at a date as in this case, the date where cumulative
exposure exceeds 50 exposure-years, we get the follow-up \emph{after}
the date classified as being in the new state if the exit
(\texttt{lex.Xst}) was to a state we defined as one of the
\texttt{precursor.states}:
\begin{Schunk}
\begin{Sinput}
> nicL$agehi <- nicL$age1st + 50 / nicL$exposure
> nicC <- cutLexis( data = nicL,
+                    cut = nicL$agehi,
+              timescale = "age",
+              new.state = "HiExp",
+       precursor.states = "Alive" )
> subset( nicC, id %in% 8:10 )
\end{Sinput}
\begin{Soutput}
          per     age     tfh lex.dur lex.Cst lex.Xst lex.id id icd exposure
4100 1934.246 47.9067 23.1861 21.7727   HiExp   D.oth      4  8 527        9
5    1934.246 54.7465 24.7890 22.0977   Alive   D.oth      5  9 150        0
6    1934.246 44.3314 23.0437  1.9563   Alive   HiExp      6 10 163        2
680  1936.203 46.2877 25.0000 16.2536   HiExp  D.lung      6 10 163        2
          dob  age1st   agein  ageout    agehi
4100 1886.340 24.7206 47.9067 69.6794 30.27616
5    1879.500 29.9575 54.7465 76.8442      Inf
6    1889.915 21.2877 44.3314 62.5413 46.28770
680  1889.915 21.2877 44.3314 62.5413 46.28770
\end{Soutput}
\begin{Sinput}
> summary( nicC, scale=1000 )
\end{Sinput}
\begin{Soutput}
Transitions:
     To
From    Alive HiExp D.oth D.lung  Records:  Events: Risk time:  Persons:
  Alive    39    83   279     65       466      427      10.77       466
  HiExp     0     8   216     72       296      288       4.58       296
  Sum      39    91   495    137       762      715      15.35       679
\end{Soutput}
\end{Schunk}
Note that the persons-years is the same, but that the number of events
has changed. This is because events are now defined as any transition
from alive, including the transitions to \texttt{HiExp}.

Also note that (so far) it is necessary to specify the variable with
the cutpoints in full, using only \texttt{cut=agehi} would give an error.

\subsection{Subdivision of existing states}
It may be of interest to subdivide the states following the
intermediate event according to wheter the event has occurred or
not. That is done by the argument \texttt{split.states=TRUE}.

Moreover, it will also often be of interest to introduce a new
timescale indicating the time since intermediate event. This can be
done by the argument \texttt{new.scale=TRUE}, alternatively
\texttt{new.scale="tfevent"}, as illustrated here:
\begin{Schunk}
\begin{Sinput}
> nicC <- cutLexis( data = nicL,
+                    cut = nicL$agehi,
+              timescale = "age",
+              new.state = "Hi",
+            split.states=TRUE, new.scale=TRUE,
+       precursor.states = "Alive" )
> subset( nicC, id %in% 8:10 )
\end{Sinput}
\begin{Soutput}
          per     age     tfh   Hi.dur lex.dur lex.Cst    lex.Xst lex.id id icd
4100 1934.246 47.9067 23.1861 17.63054 21.7727      Hi  D.oth(Hi)      4  8 527
5    1934.246 54.7465 24.7890       NA 22.0977   Alive      D.oth      5  9 150
6    1934.246 44.3314 23.0437       NA  1.9563   Alive         Hi      6 10 163
680  1936.203 46.2877 25.0000  0.00000 16.2536      Hi D.lung(Hi)      6 10 163
     exposure      dob  age1st   agein  ageout    agehi
4100        9 1886.340 24.7206 47.9067 69.6794 30.27616
5           0 1879.500 29.9575 54.7465 76.8442      Inf
6           2 1889.915 21.2877 44.3314 62.5413 46.28770
680         2 1889.915 21.2877 44.3314 62.5413 46.28770
\end{Soutput}
\begin{Sinput}
> summary( nicC, scale=1000 )
\end{Sinput}
\begin{Soutput}
Transitions:
     To
From    Alive Hi D.oth D.lung D.lung(Hi) D.oth(Hi)  Records:  Events:
  Alive    39 83   279     65          0         0       466      427
  Hi        0  8     0      0         72       216       296      288
  Sum      39 91   279     65         72       216       762      715
       
Transitions:
     To
From    Risk time:  Persons:
  Alive      10.77       466
  Hi          4.58       296
  Sum        15.35       679
\end{Soutput}
\end{Schunk}

\section{Multiple events of the same type (recurrent events)}
Sometimes more events of the same type are recorded for each person and
one would then like to count these and put follow-up time in states accordingly.
Essentially, each set of cutpoints represents progressions from one
state to the next. Therefore the states should be numbered, and the
numbering of states subsequently occupied be increased accordingly.

This is a behaviour different from the one outlined above, and it is
achieved by the argument \texttt{count=TRUE} to
\texttt{cutLexis}. When \texttt{count} is set to \texttt{TRUE}, the
value of the arguments \texttt{new.state} and
\texttt{precursor.states} are ignored.  Actually, when using the
argument \texttt{count=TRUE}, the function \texttt{countLexis} is
called, so an alternative is to use this directly.

\end{document}


